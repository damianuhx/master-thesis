
\newcommand{\comment}[1] 

\documentclass[doc]{apa6}
   
\usepackage{apacite}
\usepackage{apacdoc}
\usepackage{hhline,float}
\usepackage{xcolor}
\usepackage{blindtext}
\usepackage{filecontents}
\usepackage{graphicx}
\usepackage[german, english]{babel}
\usepackage{subscript}
\usepackage{tabularx}
\usepackage{appendix}
\usepackage[doublespacing]{setspace}
\usepackage{float}
\restylefloat{table}

\doublespacing

   \bibliographystyle{apacite}
   \title{\vspace{1cm}The Fear of Being Laughed at and the Interpretation of Aggressive Humor}

   \shorttitle{PLaT}
   \author{\begin{small}\vspace{1cm} Author:\\Damian Hiltebrand\\ \vspace{1cm}Supervisors:\\ M.Sc. Tracey Platt  \\ lic. phil. Jennifer Hofmann  \\ \vspace{1cm} Referee:\\Prof. Dr. Willibald Ruch \end{small}}

    
   \affiliation{\vspace{2cm}University of Zurich\\Department of Psychology\\Personality and Assessment\\April 2013}
   
\begin{document}
    
\begin{center}
    
Lizenziatsarbeit der Philosophischen Fakult\"at der Universit\"at Z\"urich

\end{center}

\maketitle

\hspace{-3.5cm}\includegraphics{lizehr}


\clearpage

\hspace{-1cm}Abstract:

Gelotophobia is a gradual dimension ranging from no to extreme fear of being laughed at. \citeauthor{A02} \citeyear{A02} reported that people high in gelotophobia state to have similar emotion profiles in friendly teasing and hostile ridicule. To test if this is due to a lack of the ability of understanding the benefits of teasing, the present study created the Perception of Laughter Test (PLaT; Hiltebrand, Platt, Ruch, \& Hofmann, 2012). It tests the skill of recognizing teasing and ridicule situations with 32 items. Teasing and ridicule stories were collected in an online survey, discussed among experts, adapted for test items and rated by experts again for their clearness. The PLaT, as well as seven accompanying inventories were filled in by 284 participants. Results showed that gelotophobes make significantly more mistakes when judging teasing items, but less mistakes when judging ridicule items. The performance in teasing items was also linked to seriousness ($r=.23$ $p<.001$) and the autism spectrum ($r=.23$, $p<.001$) while the total performance was most related to the ability of estimating someones joy ($r=.39$, $p<.001$). 

\clearpage

\tableofcontents

\clearpage

\listoffigures 

\clearpage

\listoftables

\clearpage

\section{Introduction}
\subsection{Humor}
\subsubsection{History of the term}

Laughing is omnipresent in everyday life. We laugh about funny situations or during amusing experiences. Laughing is associated with positive feelings and affiliation. Humor is often seen as synonymous with laughter. The core of the experience of humor is the perception that something is ``funny'' and indeed ratings of the degree of funniness is what is assessed most frequently in experimental humor research \cite{A52}.

However the term humor can not be defined so easily. \citeauthor{A51} \citeyear{A51} gives a summary of the development and shifts of the term over time and how it can be looked at today: Before humor was associated with funniness in the late $16^{th}$ century, ``humors'' referred to bloody fluids (i.e. blood, phlegm, black bile, and yellow bile). The mixture of those four humors is expressed in physical appearance, physiognomy and proneness to disease. Ideally all the four humors are balanced.

At the beginning of the $17^{th}$ century the meaning of the term humor shifted and described the predominant mood. So good-humored, for example, denoted a disposition, trait or habit to be cheerful. The meaning of humor was also expanded to include behavior deviating from social norms. So a humorist originally was someone with an odd character. Humor became seen as a talent to make others laugh. Emerging humanism promoted the next shift in the meaning of humor. Moralists were claiming that people should not be laughed at because of peculiarities beyond their control. Thus humor was seen to be based on a sympathetic heart \cite{A60}.

Later humor was also used as an umbrella-term for all phenomena that can be perceived as funny -- mostly in Anglo-American research. This can also include mean-spirited jokes or sarcasm. This meaning is also how it is used in everyday language. 

Today the term humor is heterogenous in scientific research and different definitions exist. There seems to be an agreement on at least two dimensions of humor: A cognitive dimension (i.e. the ability to create jokes and put things in a funny context) and an affective emotional dimension reflecting one's motivation (benevolence vs. malevolence) \cite{A51}. Humor cognition is an intellectual trait related to intelligence and creativity, whereas humor as motivational and communicative expression is related to social and temperamental variables \cite{A52}. This study will be framed within the second, affective emotional meaning of humor.


\subsubsection{Positive side of humor}

A lot of evidence exists of the benefits of humor. Humor can lower state-anxiety levels \cite{A68}. Humor contributes to emotional health and is also important in learning and in social relationships \cite{A51}. 

Humor forms part of a wide variety of social interactions \cite{A67b}. \citeauthor{A67} \citeyear{A67} state that humor is generally considered to facilitate social relationships and identified several import functions in social interactions: Humor provides possibilities to connect, eases communication, minimizes differences, and invites partners in a deeper relationship. Humor creates an environment that eases tensions, deals with the serious, and makes being fully present easier. In intimate relationships, humor helps partners to better know, understand, trust, and affirm each other. It helps to share commonalities, care for and support each other, to deal with difficult issues, and to be fully oneself. 
                              

Humor also has important functions at a societal level \cite{A52}. It can imply but also untighten social norms. For instance \citeauthor{H01} \citeyear{H01} describes the subversive, as well as the normalizing, functions of humor between emigrated and sedentary albanian family members. Humor is also a powerful tool to motivate and direct people. Evidence from \citeauthor{A69} \citeyear{A69} suggests that employee humor is associated with enhanced work performance, satisfaction, workgroup cohesion, health, and coping effectiveness, as well as decreased burnout, stress, and work withdrawal. Supervisor use of humor is associated with enhanced subordinate work performance, satisfaction, perception of supervisor performance, satisfaction with supervisor, and workgroup cohesion, as well as reduced work withdrawal. 

The broad variety of uses of humor makes it clear that humor is not always the same. In fact there are many different ``humor styles'' defined by various humor researchers. It is obvious that people differ in the use of humor styles. There exists a variety of questionnaires that assess several humor styles according to the taxonomy of the correspondent authors. Examples of humor styles are: Warm, reflective, adaptive, ironic and affiliative -- to name a few.
 

\subsection{Negative Side of Humor}

\subsubsection{Friendly and hostile laughter}

Humor is mostly regarded as something positive and associated with positive effects and emotions. However not every type of humor has positive effects or is inducing positive emotions. The taxonomy by \citeauthor{P01} \citeyear{P01} makes it clear that the term humor has positive and negative meanings. There is  a negative side of humor where someone's ugliness or stupidity is laughed at. This makes it reasonable to distinguish between friendly and hostile laughter though it is not always possible to distinctly discriminate between them. Often a context is needed to clearly tell if it is a friendly or a hostile laughter. One can also easily imagine a friendly laughter that includes a weak hostile component and vice versa. This study will only deal with laughter that is clearly friendly or hostile, also by regarding the context.

\subsubsection{Aggressive humor}

According to \citeauthor{A66} \citeyear{A66} aggressive humor is a hostile, maladaptive,                                                                                                                                                                                                                                                          interpersonal style which involves humor used to manipulate others by means of an implied threat of ridicule, the tendency to express humor without regard for its potential negative impact on others (e.g., sexist or racist humor), and compulsive expressions of humor in which one finds it difficult to resist the impulse to say funny things that are likely to hurt or alienate others. Aggressive humor involves sarcasm, ridicule, teasing, derision, ``put-down'', or disparagement humor. This humor style is described as ``katagelasticism'' by \citeauthor{A13} \citeyear{A13}.

Though compulsive comments about objectionable characteristics are an essential part of disparaging humor, not every remark of that type has to be disparaging. Imagine a wife telling her husband in a lovely voice that he should buy her earmuffs because he was snoring so loud last night, or a boy compassionately calling  his girlfriend a scatterbrain after she walked accidentally into a glass door. In these cases the actor obviously did not want to disparage the target. It is more a humorous, affectionate way of expressing things. This is also referred to as pro-social teasing. Aggressive humor can also be used to comment on something important that might have been missappreciated when telling it in a serious way. Also disparaging forms of aggressive humor like ridicule or sarcasm can be used pro-socially, e.g. by making antisocial behavior or evil-doing the butt of a joke. 

Aggressive humor can be summarized as compulsive expressions of humor in which one finds it difficult to resist the impulse to say funny things that are likely to have an effect on others \cite{A52}. This is often associated with disparaging humor but can be enhancing as well. 

\subsubsection{Disparagement humor}

Disparagement humor is a subtype of aggressive humor that is used to debase a person \cite{B02a}. Ridicule is one example of that type. Disparagement humor implies social norms and makes people act according to these social norms \cite{C04a}. This negative side of humor is a powerful tool to control behavior of others. People that experience ridicule are more conforming and have more fear of failure \cite{C04c}. It is self-evident that disparagement humor is typically not appreciated by people. No one wants to be demeaned or bullied, even if others might think it is funny to do this. 


\subsubsection{Definitions of teasing and ridicule}

Teasing and ridicule are both types of aggressive humor. \citeauthor{A17} \citeyear{A17} discerns pro-social and anti-social teasing. The present study names only the pro-social type teasing. The anti-social type is referred to as ridicule.

\citeauthor{A15} \citeyear[p.~229]{A15} define teasing as ``an intentional provocation accompanied by playful markers that together comment on something of relevance to the target of the tease''. The extension of the term teasing is heterogenous in scientific research. Other authors also include bullying and ridicule to the term of teasing (\citeNP{A17};  \citeNP{A21}; \citeNP{A20}; \citeNP{A27}). Some authors even just see the bullying hostile part in teasing (\citeNP{C01a}; \citeNP{A19}). According to \citeauthor{A15} \citeyear{A15} teasing is just the friendly form and can be distinguished from bullying by the pro-social outcome. This study will use the term teasing  as the friendly pro-social type as defined by \citeauthor{A15} \citeyear{A15}. 

Ridicule on the other hand can be seen as a subtype of bullying. Bullying is defined as the act of repeatedly humiliating a weaker person. This can be achieved either by physical intimidation or assault, or by verbal abuse that ridicules or demeans someone \cite{C06h}. So ridicule is a special type of verbal bullying, where someone is laughed at in order to demean him or her. To ridicule is referred to in the literature as laughing at someone with the prime goal to debase the target for egoistic purposes like increasing the actors status. The anti-social hostile disparaging shape of humor that is a part of bullying and has no pro-social outcome will be named ridicule in this article.


\subsection{Gelotophobia: The Fear of Being Laughed at}

\subsubsection{The original concept of gelotophobia}

Until some years ago, relatively little was known about disparagement humor and its effect on the object of disparagement laughter. While most of the literature was on negative humor styles and destructive humor behavior (e.g., ridicule), individuals with no or little sense of humor, as well as individuals with even fearful responses towards humor and laughter had received only little attention. To investigate the effects of repeated mockery on the object of derisive laughter, \citeauthor{A10} \citeyear{A10} used the concept of ``gelotophobia''. 

Gelotophobia is the fear of being laughed at and is described by \citeauthor{A01} (\citeyearNP{titze96};  \citeyearNP{titze97}; \citeyearNP{A01}). He related his work to findings by other authors like \citeauthor{A46} \citeyear{A46} who described the ``Pinnocchio Complex'':  People with fear related stiff movements resembling those from Pinocchio in the fairy tale with the same name.

\citeauthor{A01} (\citeyearNP{titze96}; \citeyearNP{titze97}; \citeyearNP{A01}) describes the appearance and its causes of people with an extreme fear of being laughed at and named the phenomena gelotophobia. He sees the origins of gelotophobia in disparagement humor. According to \citeauthor{A01} \citeyear{A01}, gelotophobia is a consequence of repeated traumatic experiences of being the target of disparagement humor. Gelotophobes are described as agelotic. This means they are not able to appreciate the benefits of laughter. Therefore, they lack liveliness, spontaneity and joy. This makes them appear cold and distant.

\citeauthor{A01} \citeyear{A01} argues that gelotophobes mostly had parents that exerted a very strong pressure on them as a child. They were forced to conform norms that have only private validity and are not shared by other people. This pressure can give rise to severe feelings of guilt and shame. Therefore he claims that gelotophobes have higher levels of shame-bound anxiety and this shame-bound anxiety results in increased self-observation in social situations. According to \citeauthor{A01} \citeyear{A01} this makes it likely that someone becomes the permanent target of mockery and this in turn causes gelotophobia. However, Titze (\citeyearNP{titze96}; \citeyearNP{titze97}; \citeyearNP{A01}) built all his work on observations and did not empirically test them.


\subsubsection{Assessment of gelotophobia}

To investigate the effects of repeated mockery on the object of derisive laughter \citeauthor{A10} \citeyear{A10} wanted to be able to assess gelotophobia in an objective way. For this reason they created a questionnaire which measures gelotophobia. It was assumed that gelotophobia is a dimension on which people differ gradually. 

According to the description by \citeauthor{A01} (\citeyearNP{titze96}; \citeyearNP{titze97}), \citeauthor{A10} \citeyear{A10} created 46 items that describe symptoms of gelotophobia. These items correspond to the description of the eight facets by \citeauthor{A01} (\citeyearNP{titze96}; \citeyearNP{titze97}): Paranoid tendency towards mockery, fear of the humor of others, critical self-consciousness of their body, critical self-consciousness of their own verbal or nonverbal communicative functions, social withdrawal, general negative response to the smiling and laughter of others, discouragement and envy when comparing with humor competence of others and traumatizing experiences in the past.


\subsubsection{Validation of gelotophobia}

It was expected that gelotophobes are more likely to agree to these items than normal controls or non-gelotophobic neurotics. Gelotophobes were hypothesized to be a subgroup of shame-based neurotics. So it was also expected that shame-based neurotics agree more to those items than non-shame-based neurotics while gelotophobes agree the most. 

To test this hypothesis \citeauthor{A10} \citeyear{A10} presented the items to participants belonging to one of these three groups and a control group. They were asked to rate each item on a four point Likert scale to what extent they agree with the described symptoms. A discriminant function analysis with the four groups as classification variable and the 46 statements as dependent variables showed one main factor indicating gelotophobia. The next strongest two factors were much weaker and were related to the subject and the difficulty of items and thus had no relevance.

As expected, the gelotophobes generally had higher mean scores than did those from the other groups. Shame-based neurotics scored higher than non-shame-based neurotics did. \citeauthor{A10} \citeyear{A10} defined cut-off points to group participants based on their scores. The mean score could allocate about two thirds of the participants to the correct group. 

Gelotophobia is related to social phobia and has common features (i.e., withdrawal from social activities due to fear). The feature that distinguishes gelotophobes from social phobes is the conviction that they think they deserve being the target of disparagement jokes. A factor analysis of people high in gelotophobia suggested that it consists of three components:  Paranoid sensitivity to anticipated ridicule, controlling the social situations and withdrawal from social activities \cite{A53}. The latter one is shared with social phobia while the former two are unique to gelotophobia. 

\subsubsection{Creation of an economic short version of the questionnaire}

In a further step \citeauthor{A14} \citeyear{A14} reduced the 46 items to 15 in order to have a more economic form to measure gelotophobia. For the short version of the questionnaire only items were chosen that experts rated as most typical for gelotophobia and had high factor loadings on the first but not on the other factors. The symptoms described in the items also had to be prevalent in the gelotophobic group but not in the group of normals. They had to discriminate between the two groups. Items had to have an item total correlation of at least $r=.40$ and the number of total items in the final test had to form a reliable test. These reductions of items formed the Geloph\flq15\frq.

\subsubsection{Cultural differences}

All these studies have been conducted in German samples, but the validity of the construct gelotophobia could be shown to be in other countries and languages as well (\citeNP{A62}; \citeNP{A04}). In almost all samples of 73 different countries the Geloph\flq15\frq \hspace{0.1mm} showed the same factorial structure. However the prevalence of gelotophiobia differed between countries. Evidence suggested that this mostly determined by the culture and not by the language of the country. Generally in collectivist cultures (i.e., most Asian countries, but not China) gelotophobia is more prevalent than in individualistic cultures (i.e., most European countries and the United States) \cite{A04}. Derived from 16 published studies in different countries (i.e., 13 european countries, China, Canada and Colombia) \citeauthor{A62} \citeyear{A62} report a mean prevalence of gelotophobia (i.e., gelotophobia score of at least 2.5) of 8.5\%. A prevalence of 11.65\% in Germany \cite{A10} and 5.38\% in Switzerland \cite{gelophswiss} was reported.


\subsection{Personality, Dispositions and Perception of Gelotophobes}

\subsubsection{Personality traits of gelotophobes}

Gelotophobia is correlated with the Big-Five-Factor-Model by \citeauthor{A30} \citeyear{A30}  and the PEN Factor Model by \citeauthor{C05e} \citeyear{C05e}. Gelotophobia correlated positively with neuroticism and negatively with extraversion (\citeNP{A16}; \citeNP{A09}). \citeauthor{A09} \citeyear{A09} found a correlation of -.53 with emotional stableness\footnote{Equates the negatively poled neuroticism scale from the EPQ-R} and -.49 with extraversion of the Big-Five-Questionnaire by \citeauthor{A30} \citeyear{A30}.  \citeauthor{A49} \citeyear{A49} report correlations of -.52 and -.49 with extraversion and emotional stableness from the Big-Five-Inventory by \citeauthor{A49b} \citeyear{A49b}. Similar correlations can also be found with the German version of the EPQ-R by \citeauthor{C05g} \citeyear{C05g}. Gelotophobia correlates with .48 with neuroticism and with -.46 with extraversion \cite{A16}. These connections were as expected from the description of gelotophobia: Gelotophobes are described as anxious and shy which is an essential part of neuroticism and as less lively and not able to appreciate laughter which is typical for introverted persons.

There are also weak negative correlations with the other three personality factors friendliness (\begin{math}r=-.29$, $\;p<.05\end{math}), consciousness (\begin{math}r=-.15\end{math}, not significant) and openness (\begin{math}r=-.21,\; p< .05\end{math}) but they did not enter a stepwise regression analysis because their effects were mostly determined by extraversion and emotional stableness  \cite{A09}. There is no correlation with psychoticism from the german version of the current EPQ-R by \citeauthor{C05g} \citeyear{C05g}. However \citeauthor{A16} \citeyear{A16} found positive correlations of gelotophobia with psychoticism from the older EPQ questionnaires from 1975 (\begin{math}r=.20$, $p<.01\end{math}) \cite{C05c}, 1972 (\begin{math}r=.32$, $p<.001\end{math}) \cite{C05b} and 1968 (\begin{math}r=.33,\;p<.001\end{math}) \cite{C05a}. This shows that gelotophobia has parallels with the original construct of psychoticism which can not be found in the revised version of the scale. Though gelotophobia is highly positively correlated with neuroticism and highly negatively with extraversion, not all of the variance of gelotophobia can be explained by these personality traits \cite{A16}.


\subsubsection{Temperamental basis of the sense of humor of gelotophobes}

The State-Trait-Cheerfulness-Inventory (STCI; \citeNP{A58}; \citeNP{A59}) is a questionnaire that assesses the temperamental basis of the sense of humor in the affective emotional sense, not in the cognitive dimension. It measures the three concepts presumed to facilitate or inhibit the release of exhilaration: Cheerfulness, seriousness, and bad mood. The three concepts represent actual (state) and habitual (trait) dispositions for lowered (cheerfulness) and enhanced (seriousness, bad mood) thresholds for the induction of exhilaration or other forms of humor behavior. Cheerfulness and bad mood have in common that they are affective concepts while seriousness denotes a quality of the frame of mind.

Gelotophobia is highly positively correlated with the trait bad mood ($r$ ranges from $.57$ to $.67$ depending on sample, \begin{math}p<.001\end{math}) and highly negatively correlated with trait cheerfulness ($r$ ranges from $-.57$ to $-.63$ depending on sample, \begin{math}p<.001\end{math}) \cite{A34}. Trait seriousness is also slightly positively correlated with gelotophobia. This correlation, however, reached significance level only in one of three samples and can also be explained by the correlation seriousness has with cheerfulness and bad mood. 

Gelotophobia is highly linked to the two affective components of the temperamental basis of humor (i.e. cheerfulness and bad mood). There is also preliminary evidence from a sample of 69 people that these components are highly correlated with the two constructs from the EPQ gelotophobia is correlated with. Cheerfulness is correlated with extraversion with \begin{math}r=.59 \: (p<.001)\end{math} and with neuroticism with \begin{math}r=-.49 \: (p<.001)\end{math} from the EPQ-RK. Bad mood is correlated with extraversion with \begin{math}r=-.51  \:(p<.001)\end{math} and with neuroticism with \begin{math}r=.76 \: (p<.001)\end{math}. Gelotophobia might help to understand parallels in personality traits and the temperamental basis of the sense of humor.

\subsubsection {Disposition to joy and fear}

As already expressed, \citeauthor{A03} \citeyear{A03} tested Titze's claim that gelotophobes do not find laughter gentil. Thus it was investigated what emotions gelotophobes compared to non-gelotophobes typically have. Gelotophobes reported generally experiencing the emotion joy as less intense and for a shorter duration, while they reported a higher intensity and duration for the feelings fear and shame. These findings are in line with the high scores in neuroticism and bad mood and the low scores in extraversion and cheerfulness. 

Joy and fear are both primary emotions: Primary emotions are innate, shared among humans and primates and have universal physiology \cite{A57}. Joy is an emotion that frees activation and generalizes readiness. Fear on the other hand is an emotion that has a protective function \cite{B03}. Fear is considered as the most important additional information gelotophobia has compared to the big five personality traits \cite{A49}. Fear can account for some of the observed behaviors such as withdrawal, protection, caution and attenuation. Fear also causes muscle tension \cite{M01} and can even lead to motor immobility in very severe cases \cite{A61}. This could explain the stiff movements gelotophobes often have.


\subsubsection{Disposition to shame}

Shame is a secondary emotion. This means it requires self-awareness. \citeauthor{B01} \citeyear{B01} writes that ``shame can be defined simply as the feeling we have when we evaluate our actions, feelings, behavior, and conclude that we have done wrong; it generates a wish to hide, to disappear or even to die.'' \fullcite[p.~2]{B01}. 

Self-awareness is also needed to feel guilt and pride. Pride can be seen as the contrary of shame, while guilt is quite similar to shame. The difference between guilt and shame is, that guilt is felt when being convinced that you have done something bad to others. On the other hand you feel ashamed when you think you have not presented yourself as good as you expect yourself to. 

Shame is an objectionable feeling and consequently it restrains someone from acting. People often cover their shame with another less undesirable feeling like anger, withdrawal or even depression \cite{B01}. This suggests that shame can be responsible for mental illnesses you would not expect at a first glance (e.g., social phobia or depression).

When looking at people that are shame-prone, often similar patterns of parent-child interactions can be found. Parents of shame-prone people often were punished by complete love withdrawal. This in turn causes the association that a single mistake may cause them to be not love-worthy anymore. This leads to a global personal interpretation of criticism and hence to shame. 


\subsubsection{Experience of ridicule and teasing}

\citeauthor{A02} \citeyear{A02} investigated the feelings gelotophobes and non-gelotophobes had when imagining teasing or ridicule scenarios. The used questionnaire was the Ridicule Teasing Scenarios questionnaire (RTSq; \citeNP{A02}) that contained four distinct teasing scenarios and four distinct ridicule scenarios, as well as one ambiguous scenario.

Gelotophobes stated to have very similar feelings when imagining teasing and ridicule scenarios. Normal controls on the other hand stated to experience completely different feelings in these two types of situations. In ridicule scenarios both groups reported high levels of fear, shame, sadness and anger. Gelotophobes showed considerably higher levels of fear and shame compared to non-gelotophobes. In teasing scenarios however gelotophobes stated to have these negative feelings too, while non-gelotophobes reported to have completely different and mostly positive feelings in these situations.

This led to the assumption that gelotophobes perceive ridicule and teasing scenarios very similarly and that they have difficulties in appreciating the benefits of teasing. This however does not tell if it is an emotional or a cognitive matter. It is possible that gelotophobes have such a strong fear of being ridiculed that every laughter causes them to perceive fear so strongly that benefits of laughter, like the ease of tension and strengthen of relationship, can not be appreciated anymore. It is also possible that gelotophobes are lacking the skill to perceive what differs ridicule from teasing.

Thus it was needed to construct an ability test that assesses the ability of distinguishing teasing from ridicule. An ability test could reveal if gelotophobes are worse in telling what is a friendly teasing and what is a hostile ridicule or if this skill is not related to gelotophobia.


\subsubsection{Ridicule history}

The initial assumption was that gelotophobes fear being laughed at because they were ridiculed more often and consistently in the past due. This is, according to \citeauthor{A01} \citeyear{A01}, because the special characteristics gelotophobes have due to their parents. Several studies were conducted to test that claim. Two studies actually found a positive correlation between gelotophobia and the reported frequency of being ridiculed in the past (\citeNP{A64}; \citeNP{A02}). Others however did not find a correlation between gelotophobia and the frequency of being ridiculed (\citeNP{A29}; \citeNP{A13}). One of these studies however found a positive correlation with the reported intensity they were laughed at \cite{A13}. This higher intensity was also found by \citeauthor{A11} \citeyear{A11} and \citeauthor{A64} \citeyear{A64}. This suggests that gelotophobes remember the situations they were laughed at in childhood and adolescence as more intense. \citeauthor{A28} \citeyear{A28} observed that persons that were high in neuroticism (that is strongly connected to gelotophobia) were more likely to perceive a situation as ridicule and they also reported more situations they were ridiculed in the past. 

So it is not possible to distinctly say if the  appearance causes the mockery or if gelotophobes just could bear the mockery less effectively. It is not known if the differences in the experience of situations they are laughed at were already present in childhood or if it is due to the higher intensity of being ridiculed in the past. 


\subsubsection{Self-esteem and skills}

Gelotolophobes have a lower self-esteem than other people but no deficits in any skills are known yet. Though gelotophobes think they are not as good at producing humor as others, humor production and gelotophobia is not related \cite{A34}.  Also the score in an IQ-test is not affected by gelotophobia, but gelotophobes tend to underestimate their IQ \cite{A12}. This indicates that humor does not differ predominantly in the cognitive but rather in the affective emotional meaning. However also the emotional affective responses have a cognitive base that should be looked at.

It is also worth mentioning that low self-esteem is related to shame \cite{B04}. \citeauthor{A19} \citeyear{A19} found that the frequency of being ridiculed in childhood is related to lower self-esteem. \citeauthor{A21} \citeyear{A21} found that a higher social skill level makes people perceiving teasing scenarios more friendly and less hostile. However this social skill level was surveyed by self-report. So the effect could also be caused by the lower self-esteem.  

\citeauthor{A25} \citeyear{A25} found a very strong correlation in normal population between gelotophobia and the Autism Spectrum Quotient ($r=.62, p<.001$) measured by the short form of the Autism Spectrum Questionnaire by \citeauthor{aq01} \citeyear{aq01}. The autism spectrum is linked to domains connected with the autism spectrum such as low social and communication skills and poor imagination and is assessed by self-report.

At present there is no study yet that has investigated the relation between an emotional intelligence test and gelotophobia. It would be interesting to see if the skill in correctly perceiving someones emotions is related to gelotophobia in some kind.

\subsection{Related Constructs and the Appreciation of Aggressive Humor}

\subsubsection{Gelotophilia and katagelasticism}
 \citeauthor{A13} \citeyear{A13} observed that not all people think being laughed at is something with negative consequences. Some people also reported positive feelings when they were laughed at. Others could not remember a bad situation where they were laughed at and even denied that being laughed at could be a harmful experience. So \citeauthor{A13} \citeyear{A13} extended their study and added the constructs gelotophilia and katagelasticism. 
 
Gelotophilia refers to the joy of being laughed at. Gelotophiles actively seek and establish situations where they can make others laugh at them. They gain joy out of these situations and do not feel embarrassed when sharing embarrassing things that happened to them, when it is for the purpose of making others laugh at them. 

Katagelasticism refers to the joy of laughing at others. Katagelasticists seek and enjoy situations where they can laugh at others. They think it is normal to laugh at others and expect the target person to fight back when he does not like being the butt of a joke. 

\citeauthor{A13} \citeyear{A13} created a questionnaire measuring all the three constructs. It consists of 45 items (15 for each construct) and is called PhoPhiKat\flq45\frq. The 15 items assessing gelotophobia are those from the Geloph\flq15\frq. The questionnaire was shown to be reliable and stable. A principal component analysis suggested three main factors being consistent with the three concepts gelotophobia, gelatophilia and katagelasticism. 

Gelotophobia was negatively correlated with gelotophilia and was not correlated with katagelasticism. Katagelasticism and gelotophilia were positively correlated. These new constructs may be helpful to understand the role of actor and targets. 


\subsubsection{The appreciation of aggressive humor}

\citeauthor{A33} \citeyear{A33} tested the appreciation of cartoons as a function of gelotophobia, gelotophilia and katagelasticism. There were two types of cartoons representing aggressive or non-aggressive humor. Teasing and ridicule both are types of aggressive humor. Each cartoon was rated on two scales: funniness and aversiveness.

Higher scores in gelotophobia correlated with higher aversiveness ratings in aggressive cartoons (\begin{math}r = .19, p < .05\end{math}) whereas gelotophobia was not correlated with non-aggressive cartoons. Higher katagelasticism scores led to less aversiveness in aggressive cartoons (\begin{math}r=.34, p < .05\end{math}), but no correlation between katagelasticism and aversiveness in non-aggressive cartoons was found. 

The aversiveness ratings in aggressive cartoons were generally higher than in non-aggressive cartoons. Higher scores in katagelasticism led to higher funniness ratings in aggressive cartoons (\begin{math}r = .41, p < .05\end{math}) whereas katagelasticism was unrelated to the funniness of non-aggressive cartoons. Gelotophobia and gelotophilia had no effect on the funniness ratings of aggressive or non-aggressive humor. 



\subsubsection{The ability to distinguish ridicule from teasing}

Gelotophobes differ in the experience of friendly teasing. Aggressive humor in cartoons caused higher aversiveness and reported feelings in teasing scenarios are similar to those experienced in hostile ridicule scenarios. 
 
The origins of that difference, however, is not clear. People high in neuroticism or low in self-esteem perceive a situation more likely as ridicule (\citeNP{A28}; \citeNP{B04}). Several studies showed that gelotophobes experienced past ridicule scenarios as more intense (\citeNP{A64}; \citeNP{A11}; \citeNP{A13}). Some studies also found a higher frequency (\citeNP{A64}; \citeNP{A02}). This leads to the question if there are cognitive differences in gelotophobes causing them to judge friendly teasing as hostile ridicule. 

\citeauthor{A02} \citeyear{A02} showed that when gelotophobes imagine being in a friendly teasing scenario, the emotion profile they claim to experience is like the one of hostile ridicule. This leads to the question if this is just an emotional problem due to their higher felt intensity of being ridiculed in the past  or would they also interpret situations differently when they are not involved. 

For that reason a test has been created that consists of distinct ridicule and teasing scenarios. Gelotophobes and non-gelotophobes rate if the stories represent a ridicule or a teasing scenario.




\subsection{Research Questions}

\subsubsection{Main research question}

The main research question the created Perception of Laughter Test (PLaT; \citeNP{plat12}) will help to answer is: Do gelotophobes differ in the amount of correctly judged teasing items. Gelotophobes and non-gelotophobes are defined by a gelotophobia score in the PhoPhiKat\flq45\frq of not lower than 3.0 or not exceeding 2.0 respectively. Due to the more negative feelings gelotophobes stated in teasing scenarios \cite{A02} and the higher perceived aversiveness in aggressive humor \cite{A33}, it is expected that gelotophobes judge more teasing items incorrectly as ridicule as non gelotophobes. This leads to the following hypothesis: 

\begin{itemize}
\item H\textsubscript{0}: Gelotophobes judge as many teasing items wrong as non-gelotophobes.
\item H\textsubscript{1}: Gelotophobes judge more teasing items wrong than non-gelotophobes.
\end{itemize}

Further research questions and the corresponding hypotheses are listed in the following subsection without further descriptions. 

\subsubsection{Side research question}

Do katagelasticists differ in the amount of correctly judged ridicule items?\begin{itemize}
\item H\textsubscript{0}: People high in katagelasticism judge as many ridicule items wrong as non-katagelasticists do.
\item H\textsubscript{1}: People high in katagelasticism judge more ridicule items wrong than non-katagelasticists do.
\end{itemize}
\vspace{10mm}

\subsubsection{Correlations with gelotophobia}
Is gelotophobia positively correlated with the perceived aversiveness in teasing? \begin{itemize}
\item H\textsubscript{0}: The score in gelotophobia is not related to the perceived aversiveness in teasing items. 
\item H\textsubscript{1}: The score in gelotophobia is positively related to the perceived aversiveness in teasing items. 
\end{itemize}

Is gelotophobia positively correlated with the perception of the ridicule component in teasing?\begin{itemize}
\item H\textsubscript{0}: The score in gelotophobia is not related to perceived ridicule component in teasing items. 
\item H\textsubscript{1}: The score in gelotophobia is positively related to the perceived ridicule component in teasing items. 
\end{itemize}

Is gelotophobia correlated with the perception of the teasing component in teasing?\begin{itemize}
\item H\textsubscript{0}: The score in gelotophobia is not related to the perception of the teasing component in teasing items. 
\item H\textsubscript{1}: The score in gelotophobia is related to the perception of the teasing component in teasing items. 
\end{itemize}

Is gelotophobia correlated with the perceived funniness in teasing?\begin{itemize}
\item H\textsubscript{0}: The score in gelotophobia is not related to the perceived funniness in teasing items. 
\item H\textsubscript{1}: The score in gelotophobia is related to the perceived funniness in teasing items. 
\end{itemize}

\subsubsection{Correlations with katagelasticism}
Is katagelasticism positively correlated with the perception of the teasing component in ridicule?
\begin{itemize}
\item H\textsubscript{0}: The score in katagelasticism is not related to the teasing component in ridicule items. 
\item H\textsubscript{1}: The score in katagelasticism is negatively  related to the teasing component in ridicule items. 
\end{itemize}

Is katagelasticism positively correlated with the perceived funniness in ridicule?\begin{itemize}
\item H\textsubscript{0}: The score in katagelasticism is not related to the perceived funniness in ridicule items. 
\item H\textsubscript{1}: The score in katagelasticism is negatively  related to the perceived funniness in ridicule items. 
\end{itemize}

Is katagelasticism negatively correlated with the perception of the ridicule component in ridicule?\begin{itemize}
\item H\textsubscript{0}: The score in katagelasticism is not related to the perception of the ridicule component in ridicule items. 
\item H\textsubscript{1}: The score in katagelasticism is negatively  related to the perception of the ridicule component in ridicule items. 
\end{itemize}

Is katagelasticism positively correlated with the perceived aversiveness in ridicule?\begin{itemize}
\item H\textsubscript{0}: The score in katagelasticism is not related to the perceived aversiveness in ridicule items. 
\item H\textsubscript{1}: The score in katagelasticism is negatively  related to the perceived aversiveness in ridicule items. 
\end{itemize}

\subsubsection{Correlations with different experiences in ridicule and teasing}
Is gelotophobia positively correlated with the frequency of teasing scenarios that are experienced as ridicule?\begin{itemize}
\item H\textsubscript{0}: Gelotophobia is not related to the frequency of teasing scenarios experienced as ridicule.
\item H\textsubscript{1}: Gelotophobia is positively related to the frequency of teasing scenarios experienced as ridicule.
\end{itemize}

Is gelotophilia positively correlated with the frequency of ridicule scenarios that are experienced as teasing?
\begin{itemize}
\item H\textsubscript{0}: Gelotophilia is not related to the frequency of ridicule scenarios experienced as teasing.
\item H\textsubscript{1}: Gelotophilia is positively related to the frequency of ridicule scenarios experienced as teasing.
\end{itemize}

Is katagelasticism correlated with the frequency of ridicule scenarios that are experienced as teasing?
\begin{itemize}
\item H\textsubscript{0}: Katagelasticism is not related to the frequency of ridicule scenarios experienced as teasing.
\item H\textsubscript{1}: Katagelasticism is related to the frequency of ridicule scenarios experienced as teasing.
\end{itemize}

Besides answering these research questions it was also looked at how gelotophobia is explained best by the EPQ-RK and the STCI-T\flq60\frq by a stepwise regression analysis. 

\section{Methods}

\subsection{Participants}

A total of 284 participants with a minimum age of 18 years were recruited. 73\% of all participants were women and 27\% were men. 50.4\% of them ($N=143$) were living in Switzerland and 40.8\% ($N=116$) lived in Germany. 25 participants (8.8\%) were living in other countries, primarily in Austria. 

\subsection{Research Design}

People participated online. They were asked to fill in various questionnaires on a webpage powered by Survey Monkey. The order of the questionnaires was: 1. Demographics and miscellaneous questions, 2. RTSq, 3. PLaT, 4. EPQ-RK, 5. PhoPhiKat\flq45\frq, 6. Test of Emotional Intelligence, 7. STCI\flq60\frq, 8. Autism Spectrum Questionnaire. The items were presented in fixed order for all but one questionnaire. In the PLaT all items were shuffled randomly. About half of all people opening the link to the study only looked at the first few pages of the online survey and did not answer a suitable amount of questions. Thus this study will use and name people as participants only if they at least filled in all questionnaires up to the PhoPhiKat\flq45\frq. Thus only the TEMINT, STCI-T\flq60\frq and AQ have missing values.

\subsection{Measures} 

\subsubsection{PLaT}

The Perception of Laughter Test \cite{plat12} is a research instrument for assessing the ability to correctly judge distinct friendly or hostile aggressive humor (i.e., teasing or ridicule). The questionnaire consists of 32 items: 19 items describing distinct ridicule scenarios and 13 items describing distinct teasing scenarios. Participants rated every item if it is teasing or ridicule. 

A pilot study revealed that almost all people, even those recruited in social phobia forums and in autism related forums, are able to judge most items correctly. Hence the item difficulty is generally very low. Nevertheless Cronbach's standardized alpha was good for teasing with $\alpha=.82$. However ridicule items showed a lower difficulty and alpha of $\alpha=.51$. The lower difficulty and alpha of ridicule items compared to teasing items could have been due to the increased amount of gelotophobes in the sample and be higher in a representative sample.

To gain further information about how the situations are perceived other facets were also assessed for every item. Here is a complete list of the seven ratings that were assessed for every item:
\begin{itemize}
\item Teasing component (from one to six)
\item Ridicule component (from one to six)
\item Perceived funniness (from one to six)
\item Perceived aversiveness (from one to six)
\item Perception of the target of laughter (teasing or ridicule)
\item Perception of the participant if he was in the given situation (teasing or ridicule)
\item Intent of the laughing person (teasing or ridicule)
\end{itemize}

\subsubsection{PhoPhiKat\flq45\frq}

The PhoPhiKat\flq45\frq \cite{A14} is a 45-item questionnaire that consists of 15 items for each of the constructs gelotophobia, gelotophilia and katagelasticism. All items are positively poled and use a four point answer scale (1=strongly disagree; 2=moderately disagree; 3=moderately agree; 4 = strongly agree). All scales showed satisfying alphas with $\alpha=.88$ for gelotophobia, $\alpha=.87$ for gelotophilia and $\alpha=.82$ for katagelasticism.

\subsubsection{RTSq}

The Ridicule Teasing Scenario questionnaire (RTSq; \citeNP{rtsq10}) is an unpublished research instrument  assessing the emotions participants experience when involved in ridicule and teasing social interactions. It consists of 14 items describing a situation of one of the following types:
\begin{itemize}
\item Teasing (4 items)
\item Ridicule (4 items)
\item Laughter without social interaction (5 items)
\item An ambiguous situation that is something between teasing and ridicule (1 item)
\end{itemize}

For every item the participant rates the extents of seven emotions he would have in the given situation on a 8-point scale each. The emotions are: Joy, Sadness, Surprise, Anger, Disgust, Fear, Shame.

\subsubsection{EPQ-RK}

The short version of the Eysenck Personality Questionnaire-Revised (EPQ-RK; \citeNP{C05f}) in a German Adaptation by \citeauthor{C05g} \citeyear{C05g} assesses personality in terms of Eysenck's theory. The 50-item questionnaire contains three content scales and a lie scale. Similar to past studies, extraversion ($\alpha=.89$) and neuroticism ($\alpha=.80$) yielded good internal consistencies while they were lower for psychoticism ($\alpha=.64$) and social desirability ($\alpha=.63$).

\subsubsection{STCI-T\flq60\frq}

The trait version of the State-Trait-Cheerfulness Inventory (STCI-T; \citeNP{A58}) measures cheerfulness, seriousness, and bad mood as habitual traits. The standard trait form STCI-T\flq60\frq  contains 60 items in a four-point answer format from 1 = ``strongly disagree'' to 4 = ``strongly agree''. All scales showed good internal consistencies of at least $\alpha=.88$.

\subsubsection{AQ}

The Autism Spectrum Questionnaire (AQ; \citeNP{aq01}) in a German adaption by \citeauthor{A32} \citeyear{A32} is a questionnaire that asks about domains connected with the autism spectrum including social and communication skills and habits. Participants answer the 50 items by indicating on a 4-point scale how strongly they agree with each statement. If the answer is a slight or definitive agreement to autistic behavior, one point is added to the overall score. Thus a higher score represents a higher occurrence of autism related attributes. The internal consistency was very good, $\alpha=.92$.


\subsubsection{TEMINT}

The Test of Emotional Intelligence (TEMINT; \citeNP{temint02}) is a performance-based test of emotional intelligence. The TEMINT contains twelve situations (e.g., ``A 30-year-old female computer specialist reports: `My cat was ill. I had to take him to the surgeon. I thought I had poisoned him with insect spray.' How did this person feel in this situation?''). The feelings that have to be rated are: Dislike, anger, fear, unease, sadness, guilt, happiness, pride, affection, and surprise. 

The score is calculated from the sum of absolute differences between the participant's and the target person's (i.e., the person who experienced the given situation) ratings. Thus higher TEMINT scores indicate lower emotional intelligence. The internal consistency of the total score ($\alpha=.81$) is similar to the ones reported by \citeyear{temint02} ($\alpha=.77$), \citeauthor{temint10} \citeyear{temint10} ($\alpha=.76$) or \citeauthor{temint11} \citeyear{temint11} ($\alpha=.80$) and higher than the one found by \citeauthor{temint09} \citeyear{temint09} ($\alpha=.63$). 


\subsubsection{Mobbing history and emotional affect}

Participants where also asked about their past mobbing experiences. The question was how often they were mobbed in childhood and adolescence by each of the following types. The answer scale ranged from 1 = ``never'' to 4 = ``often''.
\begin{itemize}
\item{Physically (i.e., being hit or pushed)}
\item{Stolen or destroyed properties}
\item{Verbally (i.e., being offended or threatened)}
\item{Socially excluded}
\item{Ridiculed}
\end{itemize}

It was also asked how easily they are emotionally affected by the emotions of others. The answer format was from 1 = ``very easily'' to 4 = ``very hardly''.

\subsection{Procedure}

\subsubsection{Creation of the PLaT}

\paragraph{Finding stories}
As a first step examples of ridicule and teasing were collected. These examples had to be typical stories with as much variety as possible. Stories where collected in the internet (i.e. forums and social networks). People were asked to tell short teasing and ridicule stories by private messages or in public topics in internet forums. Topics were created in subject related German speaking forums. These were mobbing-forums (i.e. forum.mobbing.net, schueler-mobbing.de, mobbing-zentrale.de, antimobbingforum-leipzig.de, hoffnung.aktiv-forum.com),  parents'-forums (i.e. mamiweb.de, netmoms.de, vaterfreuden.de) and others (i.e. dysmorphophobie.de, psychologieforum.de, rehakids.de).
Teasing stories were more often retrieved by people that were asked individually and ridicule stories more often in forums. Teasing stories were mostly with adults and ridicule stories mostly with children and adolescents. Also the setting of teasing scenarios was more often with friends and ridicule scenarios more often in school. Eventually 97 stories were collected.

\paragraph{Selecting stories and creating items}
All stories were discussed among six experts for their suitability. 48 stories (19 teasing and 29 ridicule) were chosen. The 48 stories were transformed into stand-alone items consisting of two to four sentences. Each item was rated by four experts on teasing and ridicule components on a five-point scale. 
Only unambiguous items that every expert rated with at least 4.00 on the scale it belongs and at most 2.00 on the scale it does not belong were taken. This yielded 33 items (13 teasing and 20 ridicule items) that could be distinguished into clear categories.

Many items were stories about school-children or students. Therefore, some items were changed to working-place and family settings. The changed items were rated again to see if they still fit the criteria of being a distinct teasing or ridicule example. Some stories were varied in difficulty by making two versions of the same story -- one with an additional hint. There are two types of additional hints: A hint about the relationship between actor and target (e.g. ``his friend'') or about the feelings of the target (e.g. ``he doesn't mind others joking about that''). Some items also exist in three versions, one without a hint and one for each type of hint. Work and family settings are found in ridicule and teasing scenarios. However school items are always ridicule while all the items where the flat mate is laughing are teasing. 

\paragraph{Expert rating}

Subject-matter experts were asked to rate the teasing and ridicule components of each item. All of the 35 experts had a profession where they have to discriminate between hostile ridicule and friendly teasing. These experts were mostly researchers or therapeutists on mobbing-related topics or school psychologists. They rated all items on two 6-point scales: The extent of teasing and the extent of ridicule. 

Only unambiguous items where the extent of ridicule was at least 1.5 higher than the extent of teasing -- or vice versa -- were evaluated in the pilot study. These were 56 items. 


\paragraph{Pilot study}

A pilot study was conducted to estimate the variance of the results in the test as well as the consistency of the items. Participants were asked to rate and say for each item if the described situation relates to teasing or to ridicule. Further they also rated in every item the extent of teasing and ridicule separately on a six-point scale.

Participants were recruited in internet forums where a higher density of gelotophobes can be expected. These are social phobia \cite{A10} or Asperger syndrome forums \cite{A25} as well as controls via the mailinglist of psychology students of the University of Zurich. In total the sample consisted of 25 controls, 48 participants from social phobia forums and 36  participants from a Asperger forums. 

Results showed that in this sample the number of correctly rated teasing items was indeed varying and participants from Asperger forums scored lower than participants from social phobia forums and they scored lower than the controls. The difference of the controls and the two experimental groups was significant at the 1\% level when applying a Mann-Whitney-$U$-Test. 

Feedbacks suggested that when a story was formed into more than one item, only one item should be used in the test as it made most participants angry to reread the almost exact same story. Some of the changed ridicule items even had a higher difficulty and a negative item-total correlation, though they were more distinct examples of the original items that all were consistent with the other items. This let the author suppose that some stories were not fully read the second time they were presented in a slightly changed wording. Thus always the original item with the fewest hints of each story was applied. 

The ratings of the chosen 13 items from all different teasing stories were highly internally consistent with a standardized Cronbach's alpha of $\alpha=.82$. The ridicule items however showed less variance in the sample. More than half of the people judged every item correctly. Hence also the internal consistency for the chosen items was much lower with a standardized Cronbach's alpha of $\alpha=.51$. As the scores in the control group were similar in teasing and ridicule it can be expected that the ridicule scale will also have a satisfying alpha in another less homogenous sample. 

An analysis of the single items showed that all items have positive item-total correlations and all but two items did not increase the internal consistency of the item pool when they are excluded. There was one teasing and one ridicule item that would have increased the standardized alpha by .01 if the item was dropped. When looking at the content of these items however there was no reason why they should be less consistent with the other items and thus they were not excluded.

\paragraph{The final test}

For the final test all but one stories were chosen. The excluded story had no variance at all in the pilot study samples and was about four years old children laughing. Many participants found it inappropriate to talk about ridicule and teasing at that age as they might not be aware of that difference.

The 32 stories (13 teasing, 19 ridicule) have to be rated if they were experienced as teasing or ridicule by the target of the laughter. To have a deeper insight also the ridicule and teasing components, as well as the perceived funniness and aversiveness have to be rated separately. The perceived funniness and aversiveness ratings can be used to compare the results of the present study with the findings of \citeauthor{A33} \citeyear{A33}. Some participants with social phobia stated that they actually thought they can discern ridicule from teasing, but they would feel a teasing as ridicule nonetheless. So it was also asked how the participant would experience the given situation if he was the one laughed at, as well as what he thinks the intent of the actor was. 

\subsubsection{Recruitment}

Participants were recruited in 86 different internet forums of various subjects (e.g., computer, television, body building, politics, finances and many more) by opening topics advertising the study. In total 87 valid cases (39 men and 48 women) where recruited in those internet forums. This group will be called ``internet forums related group'' in this study. Furthermore participants where recruited by a mailinglist for psychology students of the University of Zurich and by internet forums for psychology students. 122 participants (13 men and 109 women) were recruited in that way. This group will be named ``psychology students related group'' in this study. The author also mailed an invitation to participate to people he knew personally from whom 17 participated (six men and eleven women). These 17 participants, the internet forums related group and the psychology students related group will be combined as the ``normal population groups''.

In German speaking countries less than 10\% of all people are assumed to exceed the gelotophobia cut-off score. To make sure to include enough participants scoring high on gelotophobia, people were also recruited in forums where a higher prevalence of gelotophobia could be expected. These were forums about social phobia and related topics as well as forums about autism or Asperger syndrome respectively. In the end 28 people (eleven men and 17 women) from social phobia forums and 26 people (eight men and 22 women) from autism related forums participated. These groups will be called ``social phobia related group'' and ``autism related group'' and summarized as ``increased gelotophobia groups''.

As a reward, an individual and general feedback (regarding ethic guidelines) was offered to every participant. Psychology students additionally got a verification that they participated for two hours in a psychological study. At some universities it is part of the undergrade studies in psychology to have participated in psychological experiments for a certain amount of time.

The psychology students related group consisted mainly of female participants. But also in the other groups always more than half of the participants were female though a higher prevalence of males could be expected in internet forums of various subjects and in autism related forums. Indeed the majority of all people from those groups that started the survey were males. However as the number of female participants that filled in all questionnaires was higher it can be reasoned that females more often completed the survey when they started it. In Table \ref{desctable} age is described for all participants and for each group separately. 

\clearpage

\begin{table}[H]
	\begin{centering}
	
	\caption{Descriptives of Age Depending on where the Participants were Recruited}
	
	\begin{tabular}{lrrrrrrr}
  	\hline
	Group related to & M$_{\textit{0}}$ & M$_{d}$ & \textit{SD} & Min & Max & \textit{Sk} & \textit{K} 	  \\
  	\cline{1-8}
	All participants ($N=283$)& 28.96 &  25.00 & 11.29 & 18.00 & 79.00 & 1.62 & 2.35 \\
	Internet forums ($N=86$)& 35.05 & 31.00 & 13.40 & 18.00 & 79.00 & 0.95 & 0.31  \\ 	
	Psychology students ($N=122$)& 23.57 & 21.00 & 6.46 & 18.00 & 49.00 & 2.33 & 5.13\\ 
	private invitations ($N=17$)& 33.06 & 29.00 & 13.76 & 22.00 & 68.00 & 1.42 & 0.60 \\ 
	Social phobia ($N=28$)& 27.04 & 25.50 & 6.92 & 19.00 & 49.00 & 1.55 & 2.10 \\ 
  	Autism ($N=30$)&32.93 & 30.00 & 11.28 & 18.00 & 61.00 & 0.79 & -0.23\\\hline
\end{tabular}
\vspace{2mm}
\begin{tablenotes}
{\small\raggedright
\textit{Note.} 
M$_{\textit{0}}$ = mean.
M$_{d}$ = median.
\textit{SD} = standard deviation.
Min = lowest value.
Max = highest value.
\textit{Sk} = skewness.
\textit{K} = kurtosis.
One participant from the internet forums related group did not state his age.
Normal population groups = groups related to internet forums, psychology students or private invitations.
Increased gelotophobia groups = groups related to social phobia or autism.\\
}
\end{tablenotes}
\label{desctable}
\end{centering}
\end{table}

\subsection{Analyse Methods}

The number of mistakes in teasing and ridicule items in the pilot study were both far from being normally distributed. Hence a Mann-Whitney $U$ rank sum test will be applied when testing differences in the PLaT skills between groups. Correlations will be tested with Pearson correlations as a linear connection is always assumed.

\section{Results}

\subsection{Descriptives}

\subsubsection{Descriptives of the PhoPhiKat, STCI-T, EPQ and AQ}

The traits assessed by the PhoPhiKat\flq45\frq, STCI-T\flq60\frq, EPQ-RK and AQ showed expected means. Table \ref{traitdesc} shows the descriptives for all assessed personality traits. The descriptives for the single groups are displayed in appendix~\ref{adesc}.

\clearpage

\begin{table}[H]
	\begin{center}
	\caption{ Descriptives of Assessed Traits}
	\begin{tabular}{lrrrrrrrr}
  	\hline
  	Variable & M$_{\textit{0}}$ & M$_{d}$ & \textit{SD} & Min & Max & \textit{Sk} & \textit{K} 	 & $\alpha$ \\
  	\hline 
Gelotophobia & 2.17 & 2.07 & 0.73 & 1.00 & 4.00 & 0.54 & -0.56 & .92 \\ 
 Gelotophilia & 2.23 & 2.20 & 0.61 & 1.00 & 3.67 & 0.08 & -0.70 & .89 \\ 
 Katagelasticism & 1.88 & 1.87 & 0.47 & 1.00 & 3.93 & 0.64 & 1.06 & .84 \\ 
 Psychoticism & 3.19 & 3.00 & 2.11 & 0.00 & 10.00 & 0.86 & 0.39 & .59 \\ 
 Extraversion & 5.90 & 6.00 & 3.95 & 0.00 & 12.00 & 0.00 & -1.33 & .89 \\ 
 Neuroticism & 6.04 & 6.00 & 3.20 & 0.00 & 12.00 & -0.05 & -1.00 & .80 \\ 
 Social Desirability & 3.15 & 3.00 & 2.25 & 0.00 & 11.00 & 0.78 & 0.24 & .65 \\ 
 Cheerfulness & 2.88 & 3.00 & 0.64 & 1.20 & 4.00 & -0.40 & -0.55 & .95 \\ 
 Seriousness & 2.54 & 2.50 & 0.50 & 1.30 & 4.00 & 0.36 & -0.13 & .88 \\ 
 Bad Mood & 2.17 & 2.15 & 0.63 & 1.10 & 3.80 & 0.40 & -0.60 & .95 \\ 
 Autism Spectrum Quotient & 19.82 & 17.00 & 10.20 & 5.00 & 49.00 & 1.09 & 0.36 & .92 \\
 \hline
\end{tabular}
\vspace{2mm}
\begin{tablenotes}
{\small\raggedright
\textit{Note.} 
$N=284$ for gelotophobia, gelotophilia, katagelasticism, psychoticism, extraversion, neuroticism and social desirability.
$N=278$ for cheerfulness, seriousness and bad mood.
$N=273$ for Autism Spectrum Quotient.
M$_{\textit{0}}$ = mean.
M$_{d}$ = median.
\textit{SD} = standard deviation.
Min = lowest value.
Max = highest value.
\textit{Sk} = skewness.
\textit{K} = kurtosis.
$\alpha$ = Cronbach's standardized alpha. \\
}
\end{tablenotes}
\label{traitdesc}
\end{center}
\end{table} 

The recruitment strategy showed to be highly successful as it recruited a high percentage of gelotophobes (29.2\%, i.e., 83 out of 284 exceeded the cut off score of 2.5) and 70.8\% with unsuspicious gelotophobic values. A closer look revealed that these were 38 (14.1\%) slight, 22 pronounced (9.2\%) and 16 extreme (6.0\%) gelotophobes. The latter two were subsumed as high gelotophobes. The high percentage of gelotophobes was mostly due to the increased gelotophobia groups. 41 slight, pronounced and extreme gelotophobes (49.4\% of all gelotophobes) were from those groups while 42 (50.6\% of all gelotophobes) were from the normal population groups. 

The increased gelotophobia groups showed a higher mean of gelotophobia scores compared to the normal population groups. It was highest in the social phobia related group ($M=2.90, SD=0.64$) followed by the autism related group ($M=2.68, SD=0.73$). The means of the social phobia related group ($t(252)=5.270$, $p<.001$) and the autism related group ($t(254)=6.861$, $p<.001$) were significantly higher than in the normal population groups. In the internet forums related group ($M=2.18, SD=0.79$), private invitations group ($M=1.96, SD=0.59$) and psychology students related group ($M=1.90, SD=0.50$) the mean was in the range of what was expected to find in normal population due to previous studies.

The other scales differed in a weaker way and could be explained by the differences in gelotophobia, with two exceptions: The mean of seriousness ($M=3.15, SD=0.50$) and Autism Spectrum Quotients ($M=38.48, SD=8.67$) was considerably highest in the autism related group. Hence attention was paid to that fact when investigating connections to seriousness and the autism spectrum: Correlations were also computed for the non autism related groups to make sure the connection is not only caused by the group membership. As the reported correlations with seriousness and the Autism Spectrum Quotient after excluding the non autism related groups did not fundamentally change, all groups will be reported as one. There were no other mentionable differences between groups in the described questionnaires.
 
 %%%%%%%%%%%%%%%%%%%%%%%
 %%%%%%%%%%%%%%%%%%

\subsubsection{Descriptives of the PLaT}

Table \ref{platdesc} describes the psychometric properties of the variables assessed by the PLaT. 

\clearpage

\begin{table}[H]
	\begin{center}
	\caption{ Descriptives for PLaT Teasing and Ridicule Items}
	\begin{tabular}{llrrrrrrrr}
  	\hline
  	 & Variable & M$_{\textit{0}}$ & M$_{d}$ & \textit{SD} & Min & Max & \textit{Sk} & \textit{K} 	 &$\alpha$ \\
  	\hline 
		\multicolumn{9}{l}{{Teasing items}}  \\
		&Number of Mistakes & 0.54 & 0.00 & 1.10 & 0.00 & 11.00 & 4.59 & 32.88 & .71 \\ 
 &Participant would feel ridicule & 1.45 & 1.00 & 1.73 & 0.00 & 11.00 & 1.99 & 6.00 & .69 \\ 
 &Goal of actor is ridicule & 0.55 & 0.00 & 1.15 & 0.00 & 12.00 & 4.86 & 37.34 & .72 \\ 
 &Mean teasing component & 5.08 & 5.23 & 0.74 & 2.00 & 6.00 & -1.45 & 2.41 & .89 \\ 
 &Mean ridicule component & 1.66 & 1.54 & 0.52 & 1.00 & 4.62 & 1.72 & 4.56 & .81 \\ 
 &Mean funniness rating & 3.60 & 3.54 & 1.03 & 1.23 & 5.92 & -0.03 & -0.71 & .91 \\ 
 &Mean aversiveness rating & 1.62 & 1.46 & 0.56 & 1.00 & 4.77 & 1.90 & 5.67 & .83 \\ 

	\multicolumn{9}{l}{{Ridicule items}}  \\
	&Number of Mistakes & 1.77 & 1.00 & 2.12 & 0.00 & 16.00 & 2.15 & 7.69 & .75 \\ 
 &Participant would feel teasing & 0.87 & 0.00 & 1.71 & 0.00 & 16.00 & 4.02 & 24.69 & .82 \\ 
 &Goal of actor is teasing & 4.31 & 4.00 & 3.30 & 0.00 & 18.00 & 0.65 & 0.26 & .77 \\ 
 &Mean teasing component & 1.68 & 1.47 & 0.69 & 1.00 & 6.00 & 2.42 & 8.46 & .91 \\ 
 &Mean ridicule component & 5.33 & 5.47 & 0.55 & 2.74 & 6.00 & -1.58 & 3.58 & .86 \\ 
 &Mean funniness rating & 1.37 & 1.21 & 0.46 & 1.00 & 5.11 & 2.81 & 15.36 & .88 \\ 
 &Mean aversiveness rating & 5.11 & 5.32 & 0.78 & 1.21 & 6.00 & -1.53 & 3.35 & .90 \\ 
	\hline
\end{tabular}
\vspace{2mm}
\begin{tablenotes}
{\small\raggedright
\textit{Note.} 
N=284.
M$_{\textit{0}}$ = mean.
M$_{d}$ = median.
\textit{SD} = standard deviation.
Min = lowest value.
Max = highest value.
\textit{Sk} = skewness.
\textit{K} = kurtosis.
$\alpha$ = Cronbach's standardized alpha. 
}
\end{tablenotes}
\label{platdesc}
\end{center}
\end{table}


Table~\ref{platdesc} shows that contrary to the pilot study, the teasing items were more often answered correctly than the ridicule items. The median of mistakes in teasing items is zero which means that more than half of all participants judged every of the 13 items correctly. The distribution of the number of mistakes ($M=0.54$, $SD=1.10$) showed that in the main those who made mistakes only made a few.

The mean of mistakes in teasing within the normal population groups is lower than in the increased gelotophobia groups. The highest mean of mistakes was found in the autism related group ($M=1.13$, $SD=1.74$) followed by the social phobia related group ($M=0.71$, $SD=0.85$). The means for the internet forums related group ($M=0.46$, $SD=0.83$), the psychology students related group ($M=0.43$, $SD=1.11$) and the private invitations group ($M=0.47$, $SD=0.87$) are remarkably lower. A Mann-Withney rank sum test showed that the rank differences between the normal population groups and the social phobia related group ($U=2452$, $Z=-2.34$, $p<.05$) as well as the autism related group ($U=2666$, $Z=-2.30$, $p<.05$) were significant.

The means of mistakes in ridicule items within the social phobia group ($M=1.43$, $SD=1.91$), the autism related group ($M=1.20$, $SD=1.61$) and the private invitations group ($M=1.41$, $SD=2.00$) were lower than in the internet forums related group ($M=2.13$, $SD=2.32$) and the psychology students related group ($M=1.90, SD=2.35$). However neither the social phobia group ($U=2662$, $Z=-1.41$, $p=.16$) nor the autism related group ($U=1674$, $Z=-1.93$, $p=.05$) differed significantly from the normal population groups. The related ratings -- these are the teasing and ridicule components as well as the funniness and aversiveness ratings -- differed accordingly to the mistakes in ridicule and teasing items. 

It was also tested if there were any items that were significantly more often judged wrong in any of the groups. It could be imaginable that some items are more often judged differently by a certain group, for instance due to thematic properties. Therefore, an unifactorial ANOVA was computed with the group membership as independent variable and the ratings in every ridicule and teasing item as dependent variables. Results showed that two of the 32 items were significantly affected by the group membership: Teasing item number seven ($F[4,279]=2.764$, $p<.05$) and ridicule item number nine ($F[4,279]=3.600$, $p<.01$). 

Teasing item number seven was a story about a girls best friend making a joke about her blushing when she sees the boy she is in love with. It was a very easy one that only three of all 284 participants judged incorrectly. Two of them were from the autism related group which is significantly more than than the other groups ($\chi^2 (1)=10.82$, $p<.05$).

The other item that showed significant differences between groups was a story about a student making an other student angry on purpose by disturbing him repeatedly when working. It was an item with a very high difficulty of 38.3\% wrong answers. People from the normal population groups judged this item significantly more often incorrect (42.3\%) than the increased gelotophobia groups (20.4\%), $\chi^2 (1)=11.79$, $p<.001$. As the increased gelotophobia groups generally made less mistakes in ridicule items, this difference could be seen as the effect of a very hard item that discriminated the skill in judging ridicule items best and, therefore, also showed groups effects compared to the other easier items. Hence no reason was seen to exclude any item due to undesired effects of the place were the participant was recruited.




\subsubsection{Age and gender in the PLaT}

Age and gender had no significant correlation with the number of mistakes in teasing or in ridicule items. Thus no gender or age effects influenced the skills the PLaT assessed.

However, age was negatively correlated with the number of ridicule items where the goal of the actor is perceived as teasing ($r=-.13$, $p<.05$). Some of the perceived components or funniness and aversiveness ratings were also correlated with age: Age ($N=283$) was correlated with the perceived teasing component in teasing items ($r=-.20$, $p<.001$) and ridicule components in teasing ($r=.16$, $p<.01$) and ridicule ($r=.13$, $p<.05$) items, as well as with aversiveness ($r=-.15$, $p<.01$) in teasing items. 

Female gender ($N=284$) showed weak positive correlations with the ridicule component ($r=.15, p<.01$) and aversiveness ratings ($r=.16$, $p<.01$) in ridicule items and the teasing component ($r=.12, p<.05$) and funniness ratings ($r=.13$, $p<.05$) in teasing items. Last but not least female gender correlated ($r=.24$, $p<.001$) with the funniness ratings in ridicule items.

\subsubsection{Internal consistency of the PLaT}

The teasing items in the PLaT showed an acceptable to good internal consistency. It was also looked at the item-total correlations. The rating (i.e., if the item relates to ridicule or teasing) in every single teasing item was positively correlated with the total of ratings in the other teasing items. Only one teasing item (i.e., teasing item number 13) showed a notably lower item-total correlation and had increased the internal consistency of the items if it was dropped. It was a teasing item where a driving teacher is making jokes about his student's problems when driving away uphill. It was the only teasing item where a superior (i.e., teacher) is laughing at a inferior (i.e., student). 

As the variance in teasing items was very low, also the internal consistencies were fluctuating between groups. While the psychology students related group and the autism related group are showing acceptable to good internal consistencies with $\alpha=.73$ and $\alpha=.72$ the other groups all have alphas below $\alpha=.50$ with $\alpha=.42$ for the internet forums related group and $\alpha=.22$ for the social phobia related group. This very low alpha in the latter group was caused by only two participants that each judged two items wrong that were all judged correctly by those that only made one mistake. As the rating of the teasing and ridicule components, as well as the funniness and aversiveness ratings, showed acceptable alphas in all groups, it can be reasoned that the unsatisfying alphas in some groups are caused by the low variances and not by items that are not adequate for some groups.

The rating of the ridicule items (i.e., if the item relates to ridicule or teasing) in the PLaT showed a good internal consistency. All but one item had a high item-total correlation and had not increased the internal consistency of ridicule items if it was dropped. The item with a lower item-total correlation was the  ridicule item number seven. It is about a teacher laughing at his student not being able to translate french words. When looking at internal consistencies for different groups all groups show similar internal consistencies for ridicule items. 

\subsubsection{Descriptives of the RTSq}

The psychometric properties of the emotions participants stated to have in the given ridicule, teasing and other scenarios without social interaction are described in Table \ref{rtsqdesc}. 

\clearpage

\begin{table}[H]
	\begin{center}
	\caption{ Descriptives of the RTSq}
	\begin{tabular}{llrrrrrrrr}
  	\hline
  	&Variable & M$_{\textit{0}}$ & M$_{d}$ & \textit{SD} & Min & Max & \textit{Sk} & \textit{K} &$\alpha$ \\
  	\hline 
	\multicolumn{9}{l}{{Teasing scenarios}}  \\	
&Joy & 3.49 & 3.25 & 1.74 & 1.00 & 7.75 & 0.34 & -0.85 & .79 \\ 
 &Sadness & 2.91 & 2.75 & 1.64 & 1.00 & 8.00 & 0.76 & -0.08 & .78 \\ 
 &Anger & 3.34 & 3.25 & 1.66 & 1.00 & 8.00 & 0.43 & -0.63 & .74 \\ 
 &Disgust & 2.66 & 2.25 & 1.62 & 1.00 & 7.00 & 0.79 & -0.36 & .75 \\ 
 &Surprise & 4.09 & 4.25 & 1.53 & 1.00 & 8.00 & 0.00 & -0.57 & .62 \\ 
 &Shame & 3.64 & 3.50 & 1.91 & 1.00 & 8.00 & 0.40 & -0.79 & .80 \\ 
 &Fear & 2.94 & 2.25 & 1.96 & 1.00 & 8.00 & 0.95 & -0.11 & .85 \\ 	
\multicolumn{9}{l}{{Ridicule scenarios}}  \\	
	
&Joy & 1.90 & 1.50 & 1.08 & 1.00 & 6.25 & 1.41 & 1.80 & .62 \\ 
 &Sadness & 4.56 & 4.75 & 1.73 & 1.00 & 8.00 & -0.16 & -0.71 & .74 \\ 
 &Anger & 5.02 & 5.25 & 1.69 & 1.00 & 8.00 & -0.17 & -0.77 & .73 \\ 
 &Disgust & 3.82 & 3.75 & 1.83 & 1.00 & 8.00 & 0.19 & -1.00 & .75 \\ 
 &Surprise & 3.97 & 4.00 & 1.57 & 1.00 & 8.00 & 0.04 & -0.59 & .64 \\ 
 &Shame & 4.53 & 4.50 & 1.91 & 1.00 & 8.00 & -0.03 & -0.84 & .79 \\ 
 &Fear & 4.36 & 4.25 & 2.01 & 1.00 & 8.00 & 0.10 & -0.93 & .81 \\ 

\multicolumn{9}{l}{{Scenarios without social interaction}}  \\	

&Joy & 5.70 & 6.20 & 1.68 & 1.00 & 8.00 & -0.76 & -0.18 & .86 \\ 
 &Sadness & 1.45 & 1.00 & 0.77 & 1.00 & 6.00 & 2.63 & 8.58 & .76 \\ 
 &Anger & 1.55 & 1.20 & 0.71 & 1.00 & 4.20 & 1.48 & 1.55 & .62 \\ 
 &Disgust & 1.22 & 1.00 & 0.50 & 1.00 & 4.40 & 3.28 & 12.69 & .70 \\ 
 &Surprise & 2.44 & 2.20 & 1.25 & 1.00 & 7.00 & 0.97 & 0.39 & .75 \\ 
 &Shame & 1.43 & 1.00 & 0.73 & 1.00 & 4.60 & 2.19 & 4.54 & .70 \\ 
 &Fear & 1.45 & 1.00 & 0.82 & 1.00 & 6.00 & 2.59 & 7.41 & .78 \\ 	

\hline
\end{tabular}
\vspace{2mm}
\begin{tablenotes}
{\small\raggedright
\textit{Note.} 
N=284.
$\alpha$ = Cronbach's standardized alpha. \hfill \strut \\
}
\end{tablenotes}
\label{rtsqdesc}
\end{center}
\end{table}

The groups with a higher prevalence of gelotophobia had more fear and shame and less joy in teasing scenarios. Also the ridicule scenarios and the ones without social interaction showed differences between those groups in the same feelings. Those differences however were smaller than in teasing scenarios. The internal consistencies were similar to the ones found by \citeauthor{A02} \citeyear{A02}. The descriptives for each separate group can be found in appendix~\ref{adesc}.

\subsubsection{Descriptives of the TEMINT}

The sums of absolute differences between the actual feeling of the target person and the particiapant's rating are described in Table \ref{temintdesc}. 

\clearpage

\begin{table}[H]
	\begin{center}
	\caption{ Descriptives of TEMINT Emotions}
	\begin{tabular}{lrrrrrrrr}
  	\hline
  	Variable & M$_{\textit{0}}$ & M$_{d}$ & \textit{SD} & Min & Max & \textit{Sk} & \textit{K} 	 \\
  	\hline 
	Aversive	     & 3.15 & 3.00 & 1.66 & 0.00 & 7.00 & 2.39 & -3.33 \\
	Angry 	     & 2.67 & 2.00 & 1.68 & 0.00 & 8.00 & 0.50 & -0.25 \\
	Fear		     & 4.90 & 5.00 & 2.14 & 0.00 & 12.00 & 0.50 & 0.28 \\
	Nervous        & 6.30 & 6.00 & 2.78 & 0.00 & 16.00 & 0.49 & 0.44 \\
	Sad		     & 2.22 & 2.00 & 1.37 & 0.00 & 8.00 & 1.00 & 1.88 \\
	Guilt		     & 1.18 & 1.00 & 0.89 & 0.00 & 4.00 & 0.38 & -0.27 \\
	Joy		     & 0.80 & 0.00 & 1.15 & 0.00 & 8.00 & 1.92 & 5.80 \\
	Pride	     & 2.00 & 3.00 & 1.13 & 1.00 & 6.00 & 0.78 & -0.02\\
	Affection 	     & 5.30 & 5.00 & 2.76 & 1.00 & 13.00 & 0.44 & -0.58\\
	Surprise 	     & 4.16 & 4.00 & 1.73 & 0.00 & 9.00 & 0.17 & -0.38\\
	Total 	     & 33.66 & 33.00 & 10.71 & 12.00 & 70.00 & 0.47 & 0.23 \\
	 \hline
\end{tabular}
\vspace{2mm}
\begin{tablenotes}
{\small\raggedright
\textit{Note.} 
N=278.
M$_{\textit{0}}$ = mean.
M$_{d}$ = median.
\textit{SD} = standard deviation.
Min = lowest value.
Max = highest value.
\textit{Sk} = skewness.
\textit{K} = kurtosis. \\
}
\end{tablenotes}
\label{temintdesc}
\end{center}
\end{table} 

An ANOVA revealed that the total sums of TEMINT scores do not differ significantly between groups ($F[4,273]=2.052$, $p=.09$). The mean of total scores ($M=33.66$, $SD=10.71$) was similar to those reported in past studies (\citeNP{temint09}; \citeNP{temint10}; \citeNP{temint11}). 


\subsubsection{Descriptives of mobbing history and emotional affect}

The psychometric properties for the ratings of how often the participant was mobbed by certain means is printed in Table~\ref{mobbdesc}. The answer scale ranged from 1 = ``never'' to 4 = ``often''. The psychometric properties of the question that asked how easily the participant is emotionally affected by the emotions of others is also displayed (1 =``very easily'' to 4 = ``very hardly'').

\clearpage

\begin{table}[H]
	\begin{center}
	\caption{Descriptives of Frequencies of Being Mobbed and Emotional Affect}
	\begin{tabular}{lrrrrrrr}
  	\hline
  	Variable & M$_{\textit{0}}$ & M$_{d}$ & \textit{SD} & Min & Max & \textit{Sk} & \textit{K} \\
  	\hline 
	Physically & 1.64 & 1.00 & 0.84 & 1.00 & 4.00 & 1.14 & 0.40 \\ 
 Properties & 1.58 & 1.00 & 0.82 & 1.00 & 4.00 & 1.24 & 0.64 \\ 
 Verbally & 2.45 & 2.00 & 0.96 & 1.00 & 4.00 & 0.19 & -0.92 \\ 
 Socially & 2.49 & 2.00 & 1.06 & 1.00 & 4.00 & 0.08 & -1.24 \\ 
 Ridiculed & 2.45 & 2.00 & 1.03 & 1.00 & 4.00 & 0.17 & -1.14 \\ 
 Emotional affect & 2.26 & 2.00 & 0.72 & 1.00 & 4.00 & 0.12 & -0.26 \\ 
	\hline
\end{tabular}
\vspace{2mm}
\begin{tablenotes}
{\small\raggedright
\textit{Note.} 
N=284.
M$_{\textit{0}}$ = mean.
M$_{d}$ = median.
\textit{SD} = standard deviation.
Min = lowest value.
Max = highest value.
\textit{Sk} = skewness.
\textit{K} = kurtosis.
Emotional affect is negatively poled.\\
}
\end{tablenotes}
\label{mobbdesc}
\end{center}
\end{table}

Table~\ref{mobbdesc} shows that the means of the subjectively remembered frequencies of being verbally insulted, socially excluded or ridiculed are all between 2.45 and 2.49 which is near the exact middle of the scale. The means were lower for physically being hit or pushed or properties being stolen or destroyed. 

All groups but the personal invitation group were tested to see if they differ significantly from each other: The lowest scores in past ridicule and (negative poled) emotional affect were within the psychology students group. Past ridicule ($F[1,207]=6.678, p<.01$) and emotional affect ($F[1,207]=17.401$, $p<.001$) scores were significantly lower than in the internet forums related group. In the internet forums related group, in turn, past ridicule ($F[1,148]=20.380$, $p<.001$) and emotional affect ($F[1,148]=17.634$, $p<.001$) were lower than in the social phobia related group that has -- and that group significantly scored lower in emotional affect ($F[1,56]=7.510$, $p<.01$) but not in past ridicule ($F[1,56]=3.132$, $p=.08$) than the autism related group. Hence reported past ridicule and emotional affect are strongly related to where the participant was recruited. 

\subsection{Hypotheses}

\subsubsection{Gelotophobia and the ability of recognizing teasing}

The main hypothesis the present study tested was if people high in gelotophobia (i.e., individuals with a gelotophobia score of at least 3.0; N=43) judge more teasing items incorrectly than people low in gelotophobia (i.e., individuals with a gelotophobia score of at most 2.0; N=136). As the number of mistakes in teasing items in the PLaT was not normally distributed, both groups entered an independent two-group Mann-Whitney $U$ Test with the sum of mistakes in teasing items as the dependent variable. The rank sums differed significantly in the expected direction ($U=2202$, $Z=-2.94$, $p<.01$). Participants high in gelotophobia had an average rank of 106.79, while participants low in gelotophobia had an average rank of  84.69. The distribution of mistakes in teasing items depending on the group is displayed in Figure \ref{h1}.

\begin{figure}[H]
\includegraphics{h1}
\caption{Distribution of mistakes in teasing items for individuals scoring high and low in gelotophobia. Light = gelotophobia score at least 3.0 ($N = 43$).
Dark = gelotophobia score at most 2.0 ($N = 136$).}
\label{h1}
\end{figure}


\subsubsection{Katagelasticism and the ability of recognizing ridicule}

The second hypothesis the study tested was if people scoring high in katagelasticism (i.e., individuals with a katagelasticism score of at least 2.5; N=26) judge more ridicule items incorrectly than people scoring low in katagelasticism (i.e., individuals with a katagelasticism score of at most 2.0; N=182). As the number of mistakes in ridicule items in the PLaT was not normally distributed, both groups entered a independent two-group Mann-Whitney $U$ Test with the sum of mistakes in ridicule items as the dependent variable. The difference of the rank sums was not significant ($U=2492$, $Z=-0.81$, $p=.24$) but was in the expected direction. Participants high in katagelasticism had a mean rank of 113.13, while participants low in katagelasticsm had a mean rank of 103.27. The distribution of mistakes in ridicule items depending on the group is displayed in Figure \ref{h2}.


\begin{figure}[H]
\includegraphics[width=\textwidth]{h2}
\caption{Distribution of mistakes in ridicule items for individuals scoring high and low in katagelasticism. Light	 = katagelasticism score at least 2.5 ($N = 26$).
Dark = katagelasticism score at most 2.0 ($N = 182$).}
\label{h2}
\end{figure}


\subsubsection{Correlations between the PhoPhiKat and the PLaT}

The test results for the hypotheses of the correlations of gelotophobia and katagelasticism with the ridicule and teasing components as well as the hypotheses about the funniness and aversiveness ratings in ridicule and teasing items are shown in Table \ref{platcor}. Note that the significance levels in the table are two-sided, except the ones for correlations testing one-sided hypotheses.

\clearpage

\begin{table}[H]
\caption{Correlations Between the PhoPhiKat\flq45\frq and the PLaT}
\label{platcor}
\begin{tabular}{ll@{\hspace{1cm}}r@{.}lr@{.}lr@{.}l}	\thickline 
&&\multicolumn{2}{c}{Gelotophobia}
&\multicolumn{2}{c}{Gelotophilia}
&\multicolumn{2}{c}{Katagelasticism}
\\ \cline{3-8}
\multicolumn{7}{l}{{Teasing items}}\\
&Mistakes &   & 13* & - & 19** & - & 10 \\ 
&Participant would feel ridicule &   & 21*** & - & 20*** & - & 02 \\ 
&Goal of actor is ridicule &   & 09 & - & 10 & - & 07 \\ 
&Teasing component & - & 12* &   & 32*** &   & 19** \\ 
&Ridicule component &   & 16** & - & 14* &   & 03 \\ 
&Funniness rating & - & 11 &   & 35*** &   & 20*** \\ 
&Aversiveness rating &   & 14** & - & 15* & - & 13* \\ 
\multicolumn{7}{l}{{Ridicule items}}\\
&Mistakes & - & 16** &   & 16** &   & 14* \\ 
&Participant would feel teasing & - & 19** &   & 13* &   & 14* \\ 
&Goal of actor is teasing & - & 15* &   & 19** &   & 12* \\ 
&Teasing component & - & 08 &   & 14* &   & 14** \\ 
&Ridicule component &   & 16** & - & 09 & - & 08 \\ 
&Funniness rating & - & 07 &   & 26*** &   & 36*** \\ 
&Aversiveness rating &   & 11 & - & 09 & - & 12* \\ 
\hline 
\end{tabular} 
\vspace{2mm}
\begin{tablenotes}
{\small\raggedright
\textit{Note.} 
N=284.
\tabfnt{*}$p< .05$.
\tabfnt{**}$p< .01$.
\tabfnt{***}$p< .001$.
}
\end{tablenotes}
\end{table} 

Table~\ref{platcor} shows that gelotophobia correlated positively with the perceived ridicule components and aversiveness ratings in teasing items. Katagelasticism correlated positively with the perceived teasing components and the funniness ratings in ridicule items. It also correlated negatively with the aversiveness ratings but not with the ridicule component.

Gelotophobia correlated with the sum of teasing items that are experienced as ridicule. Katagelasticism and gelotophilia had a significant effect on the sum of ridicule items that were perceived as teasing.

Though all the directed hypotheses about gelotophobia and the perception of teasing could be proven, a closer look at Table~\ref{platcor} revealed that there are similar correlations with the same ratings in ridicule items. So gelotophobia not only affected the judgment of teasing but also of ridicule items. Furthermore the correlations for gelotophilia make it evident that gelotophilia had also been a good predictor for the research question that were formulated with gelotophobia. Unlike gelotophobia, gelotophilia was also correlated with the teasing component and funniness rating in teasing items. 


\subsubsection{Explaining gelotophobia}

In a stepwise regression analysis all assessed traits from the EPQ-RK, STCI-T\flq60\frq and the AQ where entered as possible predictor variables for gelotophobia. The results are displayed in Table \ref{step}. 

\begin{table}[H]
\caption{Predictors of Gelotophobia}
\label{step}
\begin{tabular}{lr@{.}lr@{.}lc}	\thickline 
&\multicolumn{2}{c}{\textit{B}}& \multicolumn{2}{c}{} & 95\% CI
\\ \cline{2-6}
Constant & 1 & 411*** & \multicolumn{2}{c}{}& [1.125,1.698] \\
Bad Mood & 0 & 196** & \multicolumn{2}{c}{}& [0.062, 0.330] \\
Extraversion & -0 & 057*** & \multicolumn{2}{c}{}& [-0.074, -0.039] \\
Neuroticism & 0 & 068*** & \multicolumn{2}{c}{}& [0.043, 0.092] \\
Autism Spectrum Quotient & 0 & 013** & \multicolumn{2}{c}{}& [0.006, 0.020] \\
$R^{2}$ & \multicolumn{2}{c}{}& &639 \\
$F$ & \multicolumn{2}{c}{}&118&509*** \\
\hline 
\end{tabular} 
\vspace{2mm}
\begin{tablenotes}
{\small\raggedright
\textit{Note.} 
$N = 272$.
CI = confidence interval.
\tabfnt{**}$p< .01$.
\tabfnt{***}$p< .001$.
}
\end{tablenotes}
\end{table} 

Table~\ref{step} shows that the suggested model with four predicting variables (i.e., bad mood, extraversion, neuroticism and Autism Spectrum Quotient) correlated with $r=.80$ ($p<.001$) with gelotophobia and explained 63.9\% of the variance. This is notably higher than any single variable correlated.

\subsection{Replicating Past Findings}

\subsubsection{Correlations between scales}

The present study not only tested the hypotheses between the PhoPhiKat\flq45\frq and the PLaT but also replicated correlations that where found in previous studies mentioned in the first chapter. Correlations between the PhoPhiKat\flq45\frq, the STCI-T\flq60\frq, the EPQ-RK and the AQ are shown in Table \ref{traitcor}. 

\clearpage

\begin{table}[H]
\caption{Correlations Between Personality Traits}
\label{traitcor}
\begin{tabular}{lr@{.}lr@{.}lr@{.}lr@{.}lr@{.}lr@{.}lr@{.}lr@{.}lr@{.}lr@{.}lr@{.}l}	\thickline 
&\multicolumn{2}{c}{Pho}
&\multicolumn{2}{c}{Phi}
&\multicolumn{2}{c}{Kat}
&\multicolumn{2}{c}{P}
&\multicolumn{2}{c}{E}
&\multicolumn{2}{c}{N}
&\multicolumn{2}{c}{C}
&\multicolumn{2}{c}{S}
&\multicolumn{2}{c}{BM}
\\ 
\cline{2-19}
Phi & - & 53*** & \multicolumn{16}{l}{} \\ 
Kat & - & 15* &   & 43***  &    \multicolumn{14}{l}{} \\ 
P & - & 04 &   & 14* &   & 20*** & \multicolumn{12}{l}{} \\ 
E & - & 65*** &   & 64*** &   & 18** &   & 03 &  \multicolumn{10}{l}{} \\ 
N &   & 64*** & - & 33*** & - & 05 &   & 02 & - & 37*** &  \multicolumn{8}{l}{} \\ 
C & - & 67*** &   & 68*** &   & 16** & - & 04 &   & 71*** & - & 55*** & \multicolumn{6}{l}{}  \\ 
S &   & 42*** & - & 49*** & - & 20*** & - & 25*** & - & 47*** &   & 24*** & - & 49*** & \multicolumn{4}{l}{}  \\ 
BM &   & 69*** & - & 44*** &   & 01 &   & 07 & - & 54*** &   & 75*** & - & 78*** &   & 39*** &    \multicolumn{2}{l}{} \\ 
AQ &   & 60*** & - & 49*** & - & 05 & - & 03 & - & 63*** &   & 36*** & - & 66*** &   & 57*** &   & 56*** \\ 
\hline 
\end{tabular} 

\vspace{2mm}

\begin{tablenotes}
{\small\raggedright
\textit{Note.} 
$N=284$ for gelotophobia, gelotophilia, katagelasticism, psychoticism, extraversion, neuroticism.
$N=278$ for cheerfulness, seriousness and bad mood.
$N=273$ for Autism Spectrum Quotient.
Pho = gelotophobia.
Phi = gelotophilia.
Kat = katagelasticism.
P = psychoticism.
E = extraversion.
N = neuroticism.
C = cheerfulness.
S = seriousness.
BM = bad mood.
AQ = Autism Spectrum Quotient.
\tabfnt{*}$p< .05$.
\tabfnt{**}$p< .01$.
\tabfnt{***}$p< .001$.}
\end{tablenotes}

\end{table} 

Table \ref{traitcor} shows that the correlations of gelotophobia with all other assessed traits were similar to those found in other studies. Also the correlation inside the PhoPhiKat\flq45\frq as well as correlations between Gelotophilia and Katagelasticism and the EPQ-RK and STCI-T\flq60\frq were as expected. In short all past findings about these traits could be substantiated.

\subsubsection{Feelings in ridicule and teasing scenarios}

\citeauthor{A02} \citeyear{A02} reported that people high in gelotophobia stated to have more fear and shame and less joy in teasing scenarios. The occurrence of these feelings was similar to those people low in gelotophobia stated to have in ridicule scenarios. In ridicule scenarios gelotophobes however stated to have these feelings even more extremely.

In this study, five additional scenarios were assessed that were not present in the study from \citeauthor{A02} \citeyear{A02}. In these scenarios someone is laughing at something that has, without a doubt, nothing to do with the person observing the laughter. These items can assess how strongly feelings are present when the participant is thinking of situations where someone is laughing about something certainly not related to him. Figure~\ref{conall} shows that also in those scenarios the differences of the extents of feelings between high and low gelotophobes differed in the same way as in the other scenarios.

\begin{figure}[H]
\includegraphics{gelortsq}
\caption{Extents of feelings depending on group and type of scenarios. Light = gelotophobia score at least 3.0 ($N = 43$). Dark = gelotophobia score at most 2.0 ($N = 136$).}
\label{conall}
\end{figure}

Mann-Whitney $U$ Tests were computed to test the significances of the differences. All three emotions (i.e., joy, shame and fear) differed in all three types of scenarios significantly at the 0.1\% level. The differences were clearest for fear ($U=486$, $Z=-8.627$) and shame ($U=486.5$, $Z=-8.244$) in teasing scenarios. However the $Z$ scores where not much lower for the differences in fear ($U=559$, $Z=-7.994$) and shame ($U=778$, $Z=-7.253$) in ridicule scenarios and fear ($U=885$, $Z=-7.968$) and shame ($U=1006$, $Z=-7.968$) in scenarios without social interaction. Joy differed less clearly in ridicule scenarios ($U=1867$, $Z=3.636$) and scenarios without social interaction ($U=1504.5$, $Z=4.800$) than in teasing scenarios ($U=1065.5$, $Z=6.285$) and also less clearly than other emotions differed.

Teasing scenarios obviously showed the biggest differences in joy between participants scoring low and those scoring high in gelotophobia. The effects of gelotophobia on fear and shame were not remarkably lower in the other scenarios

Wilcoxon signed-ranks tests were computed to test if the feelings of participants high and low in gelotophobia differed between scenarios. Low gelotophobes stated to have significantly different presences of joy ($Z=9.711$, $p<.001$), fear ($Z=-9.409$, $p<.001$) and shame ($Z=-7.795$, $p<.001$) in teasing and ridicule scenarios. Participants high in gelotophobia differed less clearly but also significantly in joy ($Z=3.629$, $p<.001$) and fear ($Z=-4.726$, $p<.001$). The difference between shame in teasing and ridicule scenarios however was not significant ($Z=-1.317$, $p=.19$). 

It was also tested if participants high or low in gelotophobia stated to have different extents of those feelings in teasing scenarios and scenarios without social interaction. The participants low in gelotophobia differed more clearly in all three emotions. This means they differed more in joy ($Z=9.124$, $p<.001$), fear ($Z=-7.200, p<.001$) and shame ($Z=-9.377$, $p<.001$) than participants high in gelotophobia differed in joy ($Z=5.479$, $p<.001$), fear ($Z=-5.582$, $p<.001$) and shame ($Z=-5.504$, $p<.001$).

\subsubsection{Mobbing history and emotional affect}

Table~\ref{mobbcor} shows the correlations between personality traits and the reported frequency of being mobbed in certain ways in the past (1 = ``never'' to 4 = ``often'') as well as how easily the participant thinks he is emotionally affected by the emotions of others (1 = ``very easily'' to 4 = ``very hardly'').

\clearpage

\begin{table}[H]
\caption{Correlations Between the Reported Frequency of Being Mobbed and Personality Traits}
\label{mobbcor}
\begin{tabular}{lr@{.}lr@{.}lr@{.}lr@{.}lr@{.}lr@{.}lr@{.}l}	\thickline 
&\multicolumn{2}{c}{Body}
&\multicolumn{2}{c}{Things}
&\multicolumn{2}{c}{Verbal}
&\multicolumn{2}{c}{Social}
&\multicolumn{2}{c}{Ridicule}
&\multicolumn{2}{c}{Affect}
\\ \cline{2-13}
Gelotophobia &   & 25*** &   & 31*** &   & 32*** &   & 43*** &   & 49*** &   & 18** \\ 
Gelotophilia & - & 12* & - & 21*** & - & 23*** & - & 38*** & - & 28*** & - & 19** \\ 
Katagelasticism & - & 10 & - & 12* & - & 03 & - & 14* & - & 11 &   & 12* \\ 
Psychoticism &   & 06 &   & 03 &   & 02 &   & 04 & - & 01 &   & 03 \\ 
Extraversion & - & 20*** & - & 24*** & - & 26*** & - & 39*** & - & 39*** & - & 36*** \\ 
Neuroticism &   & 12* &   & 22*** &   & 22*** &   & 29*** &   & 37*** & - & 06 \\ 
Cheerfulness & - & 21*** & - & 28*** & - & 31*** & - & 41*** & - & 39*** & - & 35*** \\ 
Seriousness &   & 23*** &   & 27*** &   & 26*** &   & 33*** &   & 30*** &   & 26*** \\ 
Bad mood &   & 24*** &   & 29*** &   & 33*** &   & 35*** &   & 45*** &   & 18** \\ 
Autism Quotient &   & 31*** &   & 38*** &   & 41*** &   & 53*** &   & 45*** &   & 43*** \\ 
\hline 
\end{tabular} 
\vspace{2mm}
\begin{tablenotes}
{\small\raggedright
\textit{Note.} 
$N = 284$. Reported frequencies of types of being mobbed during childhood and adolencence (1 = never, 4 = often).
Body =  physically being hit or pushed.
Things = properties destroyed or stolen.
Verbal = verbally insulted.
Social = excluded from social activities.
Ridicule = being ridiculed.
Affect = answer to the question how easily are you affected by the emotions of others.
\tabfnt{*}$p< .05$.
\tabfnt{**}$p< .01$.
\tabfnt{***}$p< .001$.}
\end{tablenotes}
\end{table} 

Gelotophobia correlated with the subjectively remembered frequency of being mobbed by ridicule in the past, $r=.49$, $p<.001$. This correlation between ridicule and gelotophobia was higher than any other correlation between past mobbing and the listed traits. Many strong correlations can also be found between most other mobbing types and the score in gelotophobia, gelotophilia, extraversion, neuroticism, cheerfulness, seriousness, bad mood and the Autism Spectrum Quotient.

The rating of how easily the participant is emotionally affected by the emotions of others showed significant correlations with gelotophobia in the expected direction. The more pronounced gelotophobia is the harder it is that they are emotionally affected by emotions of others. Extraversion, cheerfulness and the autism spectrum questionnaire however were much stronger correlated.

\subsection{Connections to Other Scales}

\subsubsection{Comparing skills}

The PLaT is an ability test assessing the skills of correctly recognizing teasing and ridicule. Thus it is likely that also other skills like emotional intelligence measured by other ability tests may be connected to these abilities. The total skill in the Test of Emotional Intelligence (TEMINT) was indeed correlated with the skill of recognizing ridicule. The correlations of the total skill as well as of the different facets separately with the skill-scores in the PLaT are displayed in Table~\ref{temintcor}.


\begin{table}[H]
\caption{Correlations Between the Skills in the TEMINT and the PLaT}
\label{temintcor}
\begin{tabular}{l@{\hspace{2cm}}r@{.}lr@{.}lr@{.}l}	\thickline 
&\multicolumn{2}{c}{Mistakes in teasing}
&\multicolumn{2}{c}{Mistakes in ridicule}
&\multicolumn{2}{c}{Mistakes in all items}
\\ \cline{2-7}
Aversive & - & 04 & & 12* & & 08 \\
Angry & - & 04 & & 07 & & 04 \\
Fear & - & 04 & & 07 & & 04 \\
Nervous &  & 10 & & 13* & & 15* \\
Sad & - & 01 & & 14* & & 12 \\
Guilty & - & 04 & & 03 & & 01 \\
Joy &   & 30*** & & 31*** & & 39*** \\
Pride &   & 04 & & 15* & & 14* \\
Affection &  & 03 &  & 15* & & 14*\\ 
Surprise & & 02 & - & 01 & & 00 \\ 
Total &  & 05 &  & 18** & & 18** \\ 
\hline 
\end{tabular} 
\vspace{2mm}

\begin{tablenotes}
{\small\raggedright
\textit{Note.} 
$N = 278$. 
\tabfnt{*}$p< .05$.
\tabfnt{***}$p< .001$.}
\end{tablenotes}
\end{table} 

When looking at the correlations of the ability scores for estimations of separate emotions in Table \ref{temintcor}, it can be seen that the skill of rating someone else's joy correlated clearly and positively with the abilities to judge teasing and ridicule correctly. These correlations are higher than those with gelotophobia or any other assessed trait -- for ridicule and teasing items. 

It is striking that the ability of estimating someone else's joy is correlating highly positively with both types of items in the PLaT. The ability of estimating someone else's joy is the only variable found in this study that correlated positively and significantly with the abilities of recognizing teasing and of recognizing ridicule. All other variables correlated in opposite directions for teasing and ridicule if they showed significant correlations with both types of items. Rank order correlations between the PLaT and the TEMINT that were also computed were not significant. This means that the correlation was caused more by some participants considerably differing from the mean than by all participants. As the medians for mistakes in judging someone else's joy (TEMINT) as well as recognizing teasing items (PLaT) both were zero, this is not surprising.

There were also significant correlations with the number of mistakes in ridicule items. However, they were all significant at the 5\% level and should not be highlighted due to the high total number of  correlations computed in this study.

\subsubsection{Questionnaires that explain the ability in the PLaT}

The ability to correctly judge teasing items correlated with gelotophia in the expected way. The correlations with the PhoPhiKat\flq45\frq as well as the TEMINT indicated that gelotophobia might not be the best predictor for the ability to correctly judge teasing or ridicule items. Thus it was also looked at how well the other assessed traits correlated with these abilities. The results are shown in Table \ref{skillcor}. 

\begin{table}[H]
\caption{Correlations Among the Skills in the PLaT and Personality Traits}
\label{skillcor}
\begin{tabular}{l@{\hspace{1cm}}r@{.}lr@{.}lr@{.}l}	\thickline 
&\multicolumn{2}{c}{Mistakes in teasing}
&\multicolumn{2}{c}{Mistakes in ridicule}
&\multicolumn{2}{c}{Mistakes in all items}
\\ \cline{2-7}
Gelotophobia &   & 13* & - & 16** & - & 07 \\ 
Gelotophilia & - & 19** &   & 16** &   & 05 \\ 
Katagelasticism & - & 10 &   & 14* &   & 07 \\ 
Psychoticism & - & 10 &   & 03 & - & 02 \\ 
Extraversion & - & 09 &   & 13* &   & 07 \\ 
Neuroticism &   & 05 & - & 02 &   & 01 \\ 
Cheerfulness & - & 19** &   & 08 & - & 02 \\ 
Seriousness &   & 23*** & - & 03 &   & 07 \\ 
Bad mood &   & 11 &   & 00 &   & 05 \\ 
Autism Quotient &   & 23*** & - & 10 &   & 01 \\ 
\hline 
\end{tabular} 
\vspace{2mm}

\begin{tablenotes}
{\small\raggedright
\textit{Note.} 
$N=284$ for gelotophobia, gelotophilia, katagelasticism, psychoticism, extraversion and neuroticism.
$N=278$ for cheerfulness, seriousness and bad mood.
$N=273$ for Autism Spectrum Quotient.
\tabfnt{*}$p< .05$.
\tabfnt{**}$p< .01$.
\tabfnt{***}$p< .001$.}
\end{tablenotes}
\end{table} 

Table \ref{skillcor} shows that seriousness and the Autism Spectrum Quotient correlated both highest with the mistakes in judged teasing items. This is considerably more than gelotophobia correlated with these constructs. Seriousness is also the only predictor variable suggested by a stepwise regression analysis with all constructs displayed in Table~\ref{skillcor} as independent variables. This means no variable explains additional significant variance to seriousness. When excluding seriousness from the list the Autism Spectrum Quotient is suggested as the only predictor variable. At this point it is also mentioned that all subscales from the AQ predicted the number of incorrectly judged teasing items in the same extent and that seriousness and that the Autism Spectrum Quotient were highly positively correlated ($r=.57, p<.001$).


A closer look at what variables from the teasing items in the PLaT is explained by seriousness reveals that all variables correlate in the same extent. The same is true for the Autism Spectrum Quotient. The correlations with the teasing component and the funniness (i.e., the positive parts) are correlated slightly but noteworthily stronger with the Autism Spectrum Quotient. 

Table~\ref{concor} shows the correlations between all assessed values from the PLaT with gelotophobia as well as the two constructs that explain the ability of recognizing teasing best (i.e., seriousness and Autism Spectrum Quotient). 

\begin{table}[H]
\caption{Correlation between the PLaT and the Best Predicting Inventory-Scales}
\label{concor}
\begin{tabular}{ll@{\hspace{1cm}}r@{.}lr@{.}lr@{.}l}	\thickline 
& &\multicolumn{2}{c}{Gelotophobia}
&\multicolumn{2}{c}{Seriousness}
&\multicolumn{2}{c}{Autism Spectrum Quotient}
\\ \cline{3-8}
\multicolumn{8}{l}{Teasing items}\\
&Mistakes &   & 13* &   & 23*** &   & 23*** \\ 
&Participant would feel ridicule &   & 21*** &   & 25*** &   & 29*** \\ 
&Goal of actor is ridicule &   & 09 &   & 20*** &   & 22*** \\ 
&Teasing component & - & 12* & - & 31*** & - & 34*** \\ 
&Ridicule component &   & 16** &   & 29*** &   & 25*** \\ 
&Funniness rating & - & 11 & - & 24*** & - & 32*** \\ 
&Aversiveness rating &   & 14* &   & 26*** &   & 22*** \\ 
\multicolumn{8}{l}{Ridicule items}\\
&Mistakes & - & 16** & - & 03 & - & 10 \\ 
&Participant would feel teasing & - & 19** &   & 02 & - & 09 \\ 
&Goal of actor is teasing & - & 15* & - & 12 & - & 19** \\ 
&Teasing component & - & 08 & - & 04 & - & 12 \\ 
&Ridicule component &   & 16** & - & 02 &   & 07 \\ 
&Funniness rating & - & 07 & - & 08 & - & 05 \\ 
&Aversiveness rating &   & 11 &   & 01 &   & 00 \\ 
\hline 
\end{tabular} 
\vspace{2mm}

\begin{tablenotes}
{\small\raggedright
\textit{Note.} 
$N=284$ for gelotophobia.
$N=278$ for seriousness.
$N=273$ for Autism Spectrum Quotient.
\tabfnt{*}$p< .05$.
\tabfnt{**}$p< .01$.
\tabfnt{***}$p< .001$.}
\end{tablenotes}
\end{table} 

The correlations between stated emotions and the ability in the PLaT can be found in Table~\ref{lastlabel}. It shows that the stated emotions joy, fear and shame in ridicule and teasing scenarios correlated with the ability in teasing or ridicule items respectively. 

\begin{table}[H]
\caption{Correlations Among Emotions and Skills in Ridicule and Teasing}
\label{lastlabel}
\begin{tabular}{ll@{\hspace{3cm}}r@{.}lr@{.}lr@{.}l}	\thickline 
&&\multicolumn{2}{c}{Mistakes in teasing}
&\multicolumn{2}{c}{Mistakes in ridicule}
&\multicolumn{2}{c}{Mistakes in all items}
\\ \cline{3-8}
\multicolumn{8}{l}{Teasing scenarios}\\
&Joy & - & 21*** &   & 15* &   & 03 \\ 
&Sadness &   & 14* & - & 15* & - & 06 \\ 
&Anger &   & 13* & - & 12* & - & 05 \\ 
&Disgust &   & 06 & - & 14* & - & 09 \\ 
&Surprise &   & 03 & - & 01 &   & 00 \\ 
&Shame &   & 17** & - & 11 & - & 02 \\ 
&Fear &   & 13* & - & 16** & - & 08 \\
\multicolumn{8}{l}{Ridicule scenarios}\\
&Joy & - & 03 &   & 27*** &   & 21*** \\ 
&Sadness &   & 06 & - & 14* & - & 09 \\ 
&Anger & - & 04 & - & 17** & - & 16** \\ 
&Disgust & - & 03 & - & 16** & - & 15* \\ 
&Surprise &   & 01 & - & 03 & - & 02 \\ 
&Shame &   & 05 & - & 15* & - & 10 \\ 
&Fear &   & 03 & - & 18** & - & 14* \\  
\multicolumn{8}{l}{Scenarios without social interaction}\\
&Joy & - & 06 &   & 12* &   & 08 \\ 
&Sadness &   & 06 & - & 18** & - & 12* \\ 
&Anger & - & 05 & - & 10 & - & 10 \\ 
&Disgust & - & 09 & - & 10 & - & 13* \\ 
&Surprise &   & 02 &   & 02 &   & 02 \\ 
&Shame &   & 01 & - & 10 & - & 08 \\ 
&Fear & - & 02 & - & 12 & - & 11 \\ 
\hline 
\end{tabular} 
\begin{tablenotes}
{\small\raggedright
\vspace{3mm}
\textit{Note.} 
$N=284$.
\tabfnt{*}$p< .05$.
\tabfnt{**}$p< .01$.
\tabfnt{***}$p< .001$.}
\end{tablenotes}
\end{table} 

\subsection{Feelings in Teasing and Ridicule}

Seriousness and the Autism Spectrum Quotient are both not concerned with explaining the appreciation of aggressive humor. The high correlations with the ability of correctly judging teasing and related components, however, suggested that a look at the feelings in teasing scenarios depending on these constructs could be meaningful.

Seriousness and the Autism Spectrum Quotient correlate with the feelings joy, fear and shame in the same way as gelotophobia does, but remarkably lower. Only for joy in teasing scenarios the Autism Spectrum Quotient is an equal predictor. Correlations with all feelings and all types of scenarios are displayed in Table~\ref{conrtsqcor}. 


\begin{table}[H]
\caption{Correlations Emotion Profiles and Selected Traits}
\label{conrtsqcor}
\begin{tabular}{ll@{\hspace{2cm}}r@{.}lr@{.}lr@{.}l}	\thickline 
&&\multicolumn{2}{c}{Gelotophobia}
&\multicolumn{2}{c}{Seriousness}
&\multicolumn{2}{c}{Autism Spectrum Quotient}
\\ \cline{3-8}
\multicolumn{7}{l}{{Teasing scenarios}}\\
&Joy & - & 38*** & - & 31*** & - & 44*** \\ 
&Sadness &   & 50*** &   & 28*** &   & 29*** \\ 
&Anger &   & 42*** &   & 23*** &   & 33*** \\ 
&Disgust &   & 34*** &   & 23*** &   & 25*** \\ 
&Surprise &   & 12* &   & 14* &   & 08 \\ 
&Shame &   & 61*** &   & 25*** &   & 34*** \\ 
&Fear &   & 67*** &   & 31*** &   & 42*** \\ 
\multicolumn{7}{l}{{Ridicule scenarios}}\\
&Joy & - & 18** & - & 15* & - & 21*** \\ 
&Sadness &   & 33*** &   & 14* &   & 07 \\ 
&Anger &   & 18** & - & 00 &   & 05 \\ 
&Disgust &   & 10 &   & 08 &   & 00 \\ 
&Surprise & - & 07 &   & 06 & - & 08 \\ 
&Shame &   & 49*** &   & 06 &   & 17** \\ 
&Fear &   & 59*** &   & 18** &   & 25*** \\ 
\multicolumn{7}{l}{{Scenarios without social interaction}}\\
&Joy & - & 13* & - & 10 & - & 10 \\ 
&Sadness &   & 23*** &   & 12* &   & 02 \\ 
&Anger &   & 10 & - & 01 & - & 01 \\ 
&Disgust &   & 01 &   & 03 & - & 09 \\ 
&Surprise & - & 09 &   & 05 &   & 01 \\ 
&Shame &   & 39*** &   & 06 &   & 20*** \\ 
&Fear &   & 51*** &   & 14* &   & 28*** \\ 
\hline 
\end{tabular} 
\vspace{2mm}

\begin{tablenotes}
{\small\raggedright
\textit{Note.} 
$N=284$ for gelotophobia. 
$N=278$ for seriousness.
$N=273$ for Autism Spectrum Quotient.
\tabfnt{*}$p< .05$.
\tabfnt{**}$p< .01$.
\tabfnt{***}$p< .001$.}
\end{tablenotes}
\end{table} 

The highest correlations can be found with gelotophobia. Figure~\ref{conteas} illustrates the distributions for people scoring high and those scoring low in the scales gelotophobia, seriousness, autism spectrum questionnaire, bad mood and cheerfulness. 

\clearpage

\begin{figure}[H]
\includegraphics[height=18cm]{conteas}
\caption{Joy, fear and shame in teasing scenarios depending on group. Light = score at least 3.0 for gelotophobia ($N=43$), seriousness ($N=50$), bad mood ($N=36$), at least 3.5 for cheerfulness ($N=36$) and at least 34 for the Autism Spectrum Quotient ($N=35$). 
Dark = score at most 2.0 for gelotophobia ($N=136$), seriousness ($N=39$), bad mood ($N=115$), cheerfulness ($N=27$) and at most 33 for autism spectrum questionnaire ($N=206$).}
\label{conteas}
\end{figure}

Figure~\ref{conteas} shows that though similar patterns can be found for all those constructs, the extent of gelotophobia is predicting the feelings joy fear and shame in teasing scenarios generally best. Fear and shame in teasing scenarios are better explained by gelotophobia than by any of the other predictors. Thus gelotophobia explains extents of fear and shame in teasing better than any other scale. However, joy seems not to be primarily related to gelotophobia.  




%%%%%%%%%%%%%%%%%%%%%%
\section{Discussion}
   
\subsection{Interpretation of Aggressive Humor}

The main assumption of this study is supported. People high in gelotophobia judge teasing significantly more often incorrectly as ridicule than people low in gelotophobia. In contrast, people high in gelotophobia  judge ridicule items more often correctly. Gelotophobia is stronger related to the direction of mistakes (i.e., if ridicule is judged as teasing or if teasing is judged as ridicule) than the combined score of both skills. Thus the proposition gelotophobes are not able to discriminate ridicule from teasing would be inappropriate. It can be better explained by a bias gelotophobes have. 

The subratings in the PLaT show that gelotophobes see more hostile components in both types of aggressive humor. This may cause them in some cases to interpret a teasing story as ridicule. On the other hand, this may also cause them to judge the ridicule stories as teasing less frequently. 

Gelotophilia correlates with the values in teasing items in opposite direction to gelotophobia. It is an even slightly better predictor for the skill in recognizing teasing. However, for recognizing ridicule, gelotophobia is predicting the skill slightly better. 

Though katagelasticism shows weak relations to ridicule items in the expected way, the connections with gelotophilia are always stronger. Except how much fun is experienced in ridicule is related strongest to katagelasticism, which is the core component of the construct. 

\subsection{Value of the PLaT}
   
The PLaT proved to be a suitable inventory to assess the skill of recognizing teasing and ridicule. Relations are found to gelotophobia, gelotophilia and katagelasticism that fit the theoretical background. The skills in the PLaT are also related to the corresponding emotions (i.e., low joy, high fear and shame) stated in ridicule and teasing scenarios. 

The internal consistency of teasing and ridicule items are between acceptable and good. The variance of mistakes in teasing items, however, is limited. This could be due to the lower amount of teasing items in two ways: First, the fewer items, the less mistakes can be made. Second, as the number of ridicule items is higher than the number of teasing items, the lower difficulty of teasing items could be due to participants' expectations: If someone expects having to answer as many ridicule items as teasing items, he might feel prompted to rate the items more often as pro-social after encountering many hostile ridicule items. 

The PLaT is a valid inventory to assess the ability of recognizing teasing and ridicule. In future, however, some teasing items should be added not only to have a higher variance -- most people in this study had every item correct -- but also to have an equal amount of teasing and ridicule items and thus prevent any effects caused by the different number of items.

In the future the author advises that purely linguistic representations of ridicule and teasing scenarios should be avoided, as they are strongly connected to the linguistic understanding of words and expressions. As the interpretation of observed situations should be assessed, movie clips could be a much better way to present such situations. In a movie the participant focuses on aspects he finds important and interprets his own observations. A text, on the other hand, defines which aspects are described in what way. At the current technical state of internet technology it is easy to present movies in an online study. Movies are also expected to be more appreciated than text by participants.

\subsection{Explaining the skills}

Exploring possible predictors revealed three unexpected variables as predictors for the skills of recognizing ridicule and teasing: Seriousness, autism spectrum and the ability to judge someone else's joy. While seriousness and the Autism Spectrum Quotient refer primarily to recognizing teasing and have minimal or no effects on the ability of recognizing ridicule, the ability to judge someone else's joy is highly positively related to both skills: Recognizing teasing and ridicule.

Seriousness is considered a cognitive component referring to the frame of mind, while the other two factors from the STCI are referring to an emotional affective component of the humorous temperament. Thus it is not surprising that the cognitive task of judging the message of humor is more related to the cognitive factor of the humorous temperament than to the emotional affective ones. 

The autism spectrum, which is also linked to seriousness, could be related by two ways  to the ability of teasing. First, some participants from the autism related group stated that it is very hard for them to imagine a situation from a text and, therefore, they think they are not good at rating the items. This is consistent with the description of autistic symptoms. Poor imagination and communication skills are important components of the domain. There are, however, also very high correlations with the stated frequency of being ridiculed in the past and Autism Spectrum Quotient. People scoring high on the autism spectrum almost always think they were ridiculed very frequently. Hence either they were really ridiculed more often or at least they more often felt or remember being ridiculed. As the past ridicule was really experienced and not read in text form, it is probably not primarily a linguistic disability causing the connection between the autism spectrum and the interpretation of teasing.

The ability of rating someone else's joy showed very high correlations to both skills: Judging teasing items and judging ridicule items. This may be surprising at a first glance, but when imagining both tasks, it becomes plausible. In the TEMINT participants rate how much joy someone experiences in a given situation. In the PLaT the task is very similar. When deciding if the situation was felt as ridicule or teasing by the person in the situation, it is mainly depending on the feelings the target person is expected to have. If someone thinks that a person feels joy when being laughed at, then it is probably teasing. When someone thinks that a person feels no joy but insulted then it is probably ridicule. So these two tasks are in fact similar and the high correlations exhibits congruent validity of the PLaT.

\subsection{Feelings of gelotophobes}   

The typical emotion pattern of gelotophobes of low joy, high fear and high shame in teasing scenarios is substantiated by the present study. It could also be shown that joy and fear change between teasing and ridicule scenarios, also for high gelotophobes. However, shame did only differ for non gelotophobes significantly. So the main effect of gelotophobia seems to be that teasing and ridicule are both provoking the same high extent of shame, while non gelotophobes feel less shame in teasing.

The scenarios without social interaction, that were not present in the study of \citeauthor{A02} \citeyear{A02}, could show that also in those situations gelotophobes feel less joy and more fear and more shame, although there is absolutely no reason why someone could fear being laughed at in these situations. Two of the eight defining facets of gelotophobia could explain why: General negative response to the smiling and laughter of others and envy when comparing with humor competence of others. To see someone laugh happily could provoke feelings of being substandard when not being cheerful and happy compared to the laughing person described in the scenario. 

Between teasing scenarios and such without social interaction also in non gelotophobes joy, fear and shame change in the same way. Hence teasing is provoking the same emotions also in non gelotophobes. The effects of teasing, which were associated with gelotophobia, are present for low gelotophobes too. This is an affirmation of the claim to see gelotophobia as a dimension with gradual differences that is present in all people to some degree. 

\subsection{Extending the Field}
   
   Aggressive humor and its meanings and functions are a complex matter. Many aspects are involved, such as cognitive components, temperamental and personality traits, societal conventions, social experience, self-esteem, empathy, emotions and many more. The PLaT is designed as an ability test where items are defined by conventions if it is a clear example of either kind. In everyday life situations often are not distinct. Usually friendly and hostile components are present in aggressive humor more equally. 
   
   It would be interesting to see the effects of gelotophobia in exactly such ambiguous situations. The bias (i.e., gelotophobes see teasing and ridicule more often as hostile) of gelotophobes indicates that they might judge most ambiguous situations as hostile, while non gelotophobes mostly judge them as friendly. An inventory consisting of ambiguous items could show how often everyday situations of aggressive humor are interpreted differently depending on the extent of gelotophobia. At present little is known about differences of people in the interpretation of aggressive humor. It would be interesting to see what causes whom to see something as friendly teasing or hostile ridicule. 
   
\subsection{Final Thoughts}

Gelotophobia could be related to past findings and a variety of other new findings to other constructs could be added. Gelotophobes have a bias towards ridicule. However, their combined skill in recognizing teasing and ridicule is not related. 

The skill in the PLaT of recognizing teasing is better explained by seriousness and the autism spectrum that both are strongly related to gelotophobia. The skill in both types of items, however, is related the most to the ability of rating someone's joy. Hence the general skill of recognizing teasing and ridicule is not explained by any personality trait but by the skill of estimating someone's joy. The emotional factors (i.e., emotion profiles and funniness and aversiveness ratings in the PLaT) are better explained by the PhoPhiKat and the STCI-T. 

As distinct ridicule and teasing examples are very easy to recognize an inventory assessing the interpretation of ambiguous aggressive humor could help to improve variance and, therefore, gain further evidence of what is causing whom to interpret a situation in a specific way. The present study found connections that give an idea of what components should be looked at. It is a pioneer work in the huge field of the interpretation of aggressive humor and personal differences. It gives a good foundation for further investigations.

\section{Summary}

Humor is not only everything that makes us laugh and entertains us, but the term has also a temperamental meaning. Humor has important functions at societal level and many studies exist that emphasize the positive effects of humor. There is however also a negative use of humor, for example when someone is laughed at in a demeaning way. 

The effects of this negative side of humor are far from being as well investigated as the positive sides. To investigate that field \citeauthor{A10} \citeyear{A10} used the construct of gelotophobia -- the fear of being laughed at -- defined by \citeauthor{A01} \citeyear{A01}. A questionnaire (i.e., PhoPhiKat\flq45\frq) was formed that assesses the extent of gelotophobia someone has, as well as the extents of gelotophilia (i.e., the joy of being laughed at) and katagelasticism (i.e., the joy of laughing at others).

\citeauthor{A02} \citeyear{A02} could show that gelotophobia plays a major role in how friendly and hostile aggressive humor (i.e., teasing and ridicule) is experienced: Gelotophobes stated to have similar feelings in friendly teasing scenarios as non gelotophobes have in hostile ridicule scenarios. Therefore, the assumptions that gelotophobes might not be able to discriminate teasing from ridicule was the main claim the present study tested.

An ability test (Perception of Laughter Test; PLaT) was created. First, 97 true stories that were experienced as teasing or ridicule were collected and discussed among experts for their suitability. From those, 47 stories were chosen and adapted for test items. These items were rated by experts (i.e., people having a profession where they have to be able to discriminate ridicule from teasing) for their clearness. A pilot study was conducted to test the suitability and consistency of the items. This yielded 13 teasing and 19 ridicule items.

The main aim of the study was to test the ability of gelotophobes of recognizing friendly teasing. To include enough gelotophobes in the sample, additional places where a higher prevalence of gelotophobia could be expected (i.e., internet forums for social phobia or autism related subjects) were used to recruit participants. Eventually 284 people (including 76 gelotophobes) filled in the PhoPhiKat\flq45\frq, the PLaT and various additional questionnaires: The STCI-T\flq60\frq (assessing the humorous temperament), the EPQ-RK (assessing personality), the AQ (assessing the Autism Spectrum Quotient), the RTSq (assessing the emotion profiles in teasing and ridicule scenarios) and the TEMINT (assessing the skill in estimating someone else's feelings). 

An item analysis of the PLaT revealed that it is internally consistent and has no undesired group effects. The variance in teasing items, however, was limited. A better result could be achieved by using additional teasing items in the future.

Findings showed that gelotophobes more often interpret friendly teasing incorrectly as hostile ridicule. In contrast, they see hostile ridicule less often incorrectly as friendly teasing. Thus it would be inappropriate to talk about a general skill and a bias explains the findings better. Katagelasticism had no significant effect on the judgment of ridicule items.

A stepwise regression analysis showed that gelotophobia is best explained by high bad mood, neuroticism and Autism Spectrum Quotient and low extraversion.

Connections between the PLaT and the other scales suggested that the ability of recognizing teasing is better explained by seriousness and the autism spectrum than by gelotophobia -- which is highly related to these constructs. The ability of recognizing ridicule however is not related to seriousness or the autism spectrum. 

The only assessed variable predicting both skills in the same direction -- the abilities of recognizing teasing and ridicule -- is the ability of judging someone else's joy assessed by the Test of Emotional Intelligence (TEMINT). As estimating someone else's joy is an important part of discriminating ridicule from teasing, this connection exhibits congruent validity of the PLaT.

Relations of gelotophobia found in past studies could be substantiated and a variety of new connections could be added. The field however is very complex and involving many parts. Hence a lot of investigation will be needed in the future to add further knowledge to the field.


\clearpage

 %%%%%%%%%%%%%%%%%%%%%%%%%  
   
\bibliography{damobib}
 
   \clearpage
   
%%%%%%%%%%%%%%%%%%%%%%%%%%%%%%%%%%% 
   
   
   
   
   \appendix
   
   \clearpage
   
   \clearpage
   
   
   
   %%%%%%%%%%%%%%%%%%%%%%%%%%%
   
\section{Additional Tables}
\label{adesc}
\addcontentsline{toc}{section}{Appendix~\ref{adesc}: Additional Tables}

\floatstyle{plaintop}
\restylefloat{table}

\begin{table}[H]
	\begin{center}
	\caption{Descriptives of Assessed Traits for the Internet Forums Related group. }
	\begin{tabular}{lrrrrrrrr}
  	\hline
  	Variable & M$_{\textit{0}}$ & M$_{d}$ & \textit{SD} & Min & Max & \textit{Sk} & \textit{K} 	 & $\alpha$ \\
  	\hline 
	Gelotophobia & 2.18 & 2.07 & 0.79 & 1.00 & 3.93 & 0.49 & -0.80 & .93 \\ 
 Gelotophilia & 2.29 & 2.27 & 0.65 & 1.00 & 3.53 & 0.08 & -0.95 & .90 \\ 
 Katagelasticism & 1.98 & 1.93 & 0.55 & 1.00 & 3.93 & 0.81 & 1.07 & .87 \\ 
 Psychoticism & 3.07 & 3.00 & 2.08 & 0.00 & 9.00 & 0.82 & 0.12 & .57 \\ 
 Extraversion & 5.85 & 6.00 & 4.05 & 0.00 & 12.00 & 0.03 & -1.40 & .89 \\ 
 Neuroticism & 5.89 & 6.00 & 3.46 & 0.00 & 12.00 & 0.05 & -1.22 & .84 \\ 
 Social Desirability & 3.10 & 3.00 & 2.30 & 0.00 & 10.00 & 0.92 & 0.31 & .68 \\ 
 Cheerfulness & 2.83 & 2.89 & 0.64 & 1.20 & 3.95 & -0.17 & -0.79 & .95 \\ 
 Seriousness & 2.51 & 2.50 & 0.48 & 1.55 & 4.00 & 0.43 & 0.22 & .86 \\ 
 Bad Mood & 2.30 & 2.30 & 0.67 & 1.15 & 3.70 & 0.20 & -0.92 & .95 \\ 
 Autism Spectrum Quotient & 20.65 & 19.00 & 9.50 & 5.00 & 47.00 & 0.87 & 0.34 & .90 \\ 
  \hline
\end{tabular}
\vspace{2mm}
\begin{tablenotes}
{\small\raggedright
\textit{Note.} 
$N=87$ for gelotophobia, gelotophilia, katagelasticism, psychoticism, extraversion, neuroticism and social desirability.
$N=83$ for cheerfulness, seriousness and bad mood.
$N=81$ for Autism Spectrum Quotient.
M$_{\textit{0}}$ = mean.
M$_{d}$ = median.
\textit{SD} = standard deviation.
Min = lowest value.
Max = highest value.
\textit{Sk} = skewness.
\textit{K} = kurtosis.
$\alpha$ = Cronbach's standardized alpha. \\
}
\end{tablenotes}
\end{center}
\end{table}


\begin{table}[H]
	\begin{center}
	\caption{Descriptives of Assessed Traits for the Psychology Students Related Group. }
	\begin{tabular}{lrrrrrrrr}
  	\hline
  	Variable & M$_{\textit{0}}$ & M$_{d}$ & \textit{SD} & Min & Max & \textit{Sk} & \textit{K} 	 & $\alpha$ \\
  	\hline 
	Gelotophobia & 1.90 & 1.83 & 0.50 & 1.00 & 3.27 & 0.43 & -0.54 & .85 \\ 
 Gelotophilia & 2.38 & 2.37 & 0.53 & 1.13 & 3.67 & 0.03 & -0.57 & .87 \\ 
 Katagelasticism & 1.87 & 1.83 & 0.37 & 1.13 & 2.87 & 0.28 & -0.27 & .74 \\ 
 Psychoticism & 2.93 & 3.00 & 1.88 & 0.00 & 10.00 & 0.92 & 0.96 & .56 \\ 
 Extraversion & 7.84 & 8.00 & 3.15 & 0.00 & 12.00 & -0.50 & -0.70 & .83 \\ 
 Neuroticism & 5.63 & 6.00 & 2.88 & 0.00 & 11.00 & -0.04 & -0.78 & .75 \\ 
 Social Desirability & 2.79 & 2.50 & 2.05 & 0.00 & 11.00 & 1.10 & 1.65 & .61 \\ 
 Cheerfulness & 3.18 & 3.20 & 0.47 & 2.00 & 4.00 & -0.39 & -0.57 & .93 \\ 
 Seriousness & 2.41 & 2.40 & 0.40 & 1.40 & 3.45 & 0.27 & 0.02 & .83 \\ 
 Bad Mood & 1.94 & 1.90 & 0.49 & 1.15 & 3.45 & 0.39 & -0.60 & .93 \\ 
 Autism Spectrum Quotient & 14.05 & 13.00 & 4.50 & 5.00 & 33.00 & 1.03 & 2.00 & .72 \\ 
  \hline
\end{tabular}
\vspace{2mm}
\begin{tablenotes}
{\small\raggedright
\textit{Note.} 
$N=122$ for gelotophobia, gelotophilia, katagelasticism, psychoticism, extraversion, neuroticism and social desirability.
$N=121$ for cheerfulness, seriousness and bad mood.
$N=120$ for Autism Spectrum Quotient.
M$_{\textit{0}}$ = mean.
M$_{d}$ = median.
\textit{SD} = standard deviation.
Min = lowest value.
Max = highest value.
\textit{Sk} = skewness.
\textit{K} = kurtosis.
$\alpha$ = Cronbach's standardized alpha. \\
}
\end{tablenotes}
\end{center}
\end{table}



\begin{table}
	\begin{center}
	\caption{Descriptives of Assessed Traits for the Private Invitations Group. }
	\begin{tabular}{lrrrrrrrr}
  	\hline
  	Variable & M$_{\textit{0}}$ & M$_{d}$ & \textit{SD} & Min & Max & \textit{Sk} & \textit{K} 	 & $\alpha$ \\
  	\hline 
	Gelotophobia & 1.96 & 1.67 & 0.59 & 1.27 & 3.27 & 0.73 & -0.81 & .89 \\ 
 Gelotophilia & 2.27 & 2.33 & 0.62 & 1.00 & 3.33 & -0.26 & -0.82 & .91 \\ 
 Katagelasticism & 1.88 & 2.00 & 0.42 & 1.27 & 2.93 & 0.59 & -0.06 & .82 \\ 
 Psychoticism & 5.18 & 4.00 & 2.56 & 2.00 & 10.00 & 0.34 & -1.27 & .65 \\ 
 Extraversion & 4.71 & 5.00 & 2.82 & 0.00 & 10.00 & 0.13 & -0.96 & .77 \\ 
 Neuroticism & 5.12 & 4.00 & 3.30 & 0.00 & 10.00 & 0.27 & -1.42 & .82 \\ 
 Social Desirability & 4.29 & 4.00 & 2.37 & 0.00 & 8.00 & 0.02 & -1.10 & .65 \\ 
 Cheerfulness & 2.88 & 2.95 & 0.65 & 1.25 & 3.90 & -0.72 & 0.21 & .99 \\ 
 Seriousness & 2.51 & 2.65 & 0.54 & 1.70 & 3.30 & -0.07 & -1.63 & .88 \\ 
 Bad Mood & 1.96 & 1.75 & 0.68 & 1.20 & 3.80 & 1.19 & 0.75 & .97 \\ 
 Autism Spectrum Quotient & 16.88 & 14.00 & 6.74 & 9.00 & 33.00 & 0.94 & -0.15 & .80 \\ 
\hline
\end{tabular}
\vspace{2mm}
\begin{tablenotes}
{\small\raggedright
\textit{Note.} 
$N=17$ for gelotophobia, gelotophilia, katagelasticism, psychoticism, extraversion, neuroticism, social desirability, cheerfulness, seriousness and bad mood.
$N=16$ for Autism Spectrum Quotient.
M$_{\textit{0}}$ = mean.
M$_{d}$ = median.
\textit{SD} = standard deviation.
Min = lowest value.
Max = highest value.
\textit{Sk} = skewness.
\textit{K} = kurtosis.
$\alpha$ = Cronbach's raw alpha. \\
}
\end{tablenotes}
\end{center}
\end{table}



\begin{table}
	\begin{center}
	\caption{Descriptives of Assessed Traits for the Social Phobia Related Group. }
	\begin{tabular}{lrrrrrrrr}
  	\hline
  	Variable & M$_{\textit{0}}$ & M$_{d}$ & \textit{SD} & Min & Max & \textit{Sk} & \textit{K} 	 & $\alpha$ \\
  	\hline 
	Gelotophobia & 2.90 & 2.97 & 0.64 & 1.80 & 3.93 & -0.39 & -1.06 & .89 \\ 
 Gelotophilia & 1.87 & 1.77 & 0.55 & 1.00 & 3.07 & 0.42 & -0.77 & .88 \\ 
 Katagelasticism & 1.76 & 1.73 & 0.50 & 1.00 & 3.13 & 0.64 & 0.36 & .89 \\ 
 Psychoticism & 3.68 & 4.00 & 2.33 & 1.00 & 9.00 & 0.72 & -0.45 & .59 \\ 
 Extraversion & 2.18 & 1.00 & 2.96 & 0.00 & 12.00 & 1.95 & 3.32 & .88 \\ 
 Neuroticism & 8.32 & 9.00 & 2.76 & 2.00 & 12.00 & -0.74 & -0.39 & .76 \\ 
 Social Desirability & 2.86 & 2.50 & 2.34 & 0.00 & 9.00 & 0.69 & -0.19 & .72 \\ 
 Cheerfulness & 2.34 & 2.26 & 0.58 & 1.32 & 3.70 & 0.00 & -0.43 & .94 \\ 
 Seriousness & 2.57 & 2.55 & 0.53 & 1.30 & 3.45 & -0.50 & -0.23 & .89 \\ 
 Bad Mood & 2.62 & 2.51 & 0.62 & 1.45 & 3.70 & 0.03 & -1.09 & .93 \\ 
 Autism Spectrum Quotient & 24.67 & 25.00 & 6.01 & 11.00 & 34.00 & -0.32 & -0.71 & .73 \\ 
 \hline
\end{tabular}
\vspace{2mm}
\begin{tablenotes}
{\small\raggedright
\textit{Note.} 
$N=28$ for gelotophobia, gelotophilia, katagelasticism, psychoticism, extraversion, neuroticism, social desirability,cheerfulness, seriousness and bad mood.
$N=27$ for Autism Spectrum Quotient.
M$_{\textit{0}}$ = mean.
M$_{d}$ = median.
\textit{SD} = standard deviation.
Min = lowest value.
Max = highest value.
\textit{Sk} = skewness.
\textit{K} = kurtosis.
$\alpha$ = Cronbach's standardized alpha. \\
}
\end{tablenotes}
\end{center}
\end{table}


\begin{table}
	\begin{center}
	\caption{Descriptives of Assessed Traits for the Autism Related Group. }
	\begin{tabular}{lrrrrrrrr}
  	\hline
  	Variable & M$_{\textit{0}}$ & M$_{d}$ & \textit{SD} & Min & Max & \textit{Sk} & \textit{K} 	 & $\alpha$ \\
  	\hline 
	Gelotophobia & 2.68 & 2.63 & 0.73 & 1.13 & 4.00 & -0.15 & -0.70 & .91 \\ 
 Gelotophilia & 1.80 & 1.80 & 0.48 & 1.07 & 3.00 & 0.30 & -0.27 & .81 \\ 
 Katagelasticism & 1.79 & 1.80 & 0.58 & 1.00 & 2.87 & 0.24 & -1.17 & .89 \\ 
 Psychoticism & 3.00 & 3.00 & 2.05 & 0.00 & 8.00 & 0.39 & -0.71 & .70 \\ 
 Extraversion & 2.27 & 2.00 & 2.41 & 0.00 & 8.00 & 1.03 & 0.07 & .77 \\ 
 Neuroticism & 6.50 & 6.50 & 3.22 & 0.00 & 12.00 & -0.26 & -0.92 & .80 \\ 
 Social Desirability & 4.37 & 4.00 & 2.22 & 0.00 & 9.00 & -0.03 & -0.75 & .64 \\ 
 Cheerfulness & 2.34 & 2.26 & 0.59 & 1.37 & 3.47 & 0.39 & -1.09 & .92 \\ 
 Seriousness & 3.15 & 3.35 & 0.50 & 2.20 & 3.75 & -0.37 & -1.41 & .89 \\ 
 Bad Mood & 2.50 & 2.50 & 0.56 & 1.10 & 3.50 & -0.43 & -0.32 & .92 \\ 
 Autism spectrum questionnaire & 38.48 & 41.00 & 8.67 & 14.00 & 49.00 & -1.46 & 1.56 & .91 \\ 
 \hline
\end{tabular}
\vspace{2mm}
\begin{tablenotes}
{\small\raggedright
\textit{Note.} 
$N=30$ for gelotophobia, gelotophilia, katagelasticism, psychoticism, extraversion, neuroticism and social desirability,.
$N=29$ for cheerfulness, seriousness, bad mood and Autism Spectrum Quotient.
M$_{\textit{0}}$ = mean.
M$_{d}$ = median.
\textit{SD} = standard deviation.
Min = lowest value.
Max = highest value.
\textit{Sk} = skewness.
\textit{K} = kurtosis.
$\alpha$ = Cronbach's standardized alpha. \\
}
\end{tablenotes}
\end{center}
\end{table}




%%%%%%%%%%%%%%%


\begin{table}
	\begin{center}
	\caption{ Descriptives of PLaT Teasing and Ridicule Items for the Internet Forums Related Group}
	\begin{tabular}{llrrrrrrrr}
  	\hline
  	 & Variable & M$_{\textit{0}}$ & M$_{d}$ & \textit{SD} & Min & Max & \textit{Sk} & \textit{K} 	 &$\alpha$ \\
  	\hline 
		\multicolumn{9}{l}{{Teasing items}}  \\
		&Number of Mistakes & 0.46 & 0.00 & 0.83 & 0.00 & 4.00 & 2.10 & 4.39 & .42 \\ 
& Participant would feel ridicule & 1.23 & 1.00 & 1.56 & 0.00 & 8.00 & 1.64 & 3.28 & .62 \\ 
& Goal of actor is ridicule & 0.56 & 0.00 & 0.87 & 0.00 & 3.00 & 1.47 & 1.22 & .34 \\ 
& Mean teasing component & 5.19 & 5.31 & 0.67 & 2.92 & 6.00 & -1.04 & 0.56 & .82 \\ 
& Mean ridicule component & 1.66 & 1.46 & 0.51 & 1.00 & 3.31 & 1.08 & 0.51 & .71 \\ 
& Mean funniness rating & 3.62 & 3.46 & 1.00 & 1.54 & 5.92 & 0.21 & -0.65 & .89 \\ 
& Mean aversiveness rating & 1.57 & 1.38 & 0.52 & 1.00 & 3.08 & 1.20 & 0.92 & .76 \\ 
	\multicolumn{9}{l}{{Ridicule items}}  \\
	&Number of Mistakes & 2.13 & 1.00 & 2.32 & 0.00 & 10.00 & 1.29 & 1.38 & .71 \\ 
&Participant would feel teasing & 0.97 & 0.00 & 1.53 & 0.00 & 7.00 & 1.82 & 2.82 & .66 \\ 
&Goal of actor is teasing & 4.48 & 4.00 & 3.48 & 0.00 & 14.00 & 0.60 & -0.21 & .79 \\ 
&Mean teasing component & 1.67 & 1.47 & 0.66 & 1.00 & 4.58 & 2.00 & 4.80 & .88 \\ 
&Mean ridicule component & 5.30 & 5.42 & 0.59 & 3.32 & 6.00 & -1.14 & 1.22 & .84 \\ 
&Mean funniness rating & 1.42 & 1.26 & 0.54 & 1.00 & 5.11 & 3.87 & 22.37 & .89 \\ 
&Mean aversiveness rating & 5.02 & 5.26 & 0.86 & 2.21 & 6.00 & -1.21 & 1.17 & .90 \\ 	\hline
\end{tabular}
\vspace{2mm}
\begin{tablenotes}
{\small\raggedright
\textit{Note.} 
N=87.
M$_{\textit{0}}$ = mean.
M$_{d}$ = median.
\textit{SD} = standard deviation.
Min = lowest value.
Max = highest value.
\textit{Sk} = skewness.
\textit{K} = kurtosis.
$\alpha$ = Cronbach's raw alpha. 
}
\end{tablenotes}
\end{center}
\end{table}

\begin{table}
	\begin{center}
	\caption{ Descriptives of PLaT Teasing and Ridicule Items for the Psychology Students Related Group}
	\begin{tabular}{llrrrrrrrr}
  	\hline
  	 & Variable & M$_{\textit{0}}$ & M$_{d}$ & \textit{SD} & Min & Max & \textit{Sk} & \textit{K} 	 &$\alpha$ \\
  	\hline 
		\multicolumn{9}{l}{{Teasing items}}  \\
		&Number of Mistakes & 0.43 & 0.00 & 1.11 & 0.00 & 11.00 & 7.27 & 65.81 & .73 \\ 
 &Participant would feel ridicule & 1.37 & 1.00 & 1.69 & 0.00 & 11.00 & 2.28 & 7.96 & .69 \\ 
 &Goal of actor is ridicule & 0.38 & 0.00 & 1.19 & 0.00 & 12.00 & 7.79 & 72.37 & .80 \\ 
 &Mean teasing component & 5.22 & 5.38 & 0.60 & 2.15 & 6.00 & -1.94 & 5.76 & .88 \\ 
 &Mean ridicule component & 1.60 & 1.54 & 0.47 & 1.00 & 4.62 & 2.55 & 12.21 & .80 \\ 
 &Mean funniness rating & 3.84 & 3.85 & 0.94 & 1.31 & 5.77 & -0.22 & -0.61 & .90 \\ 
 &Mean aversiveness rating & 1.59 & 1.54 & 0.49 & 1.00 & 4.38 & 2.16 & 8.41 & .78 \\ 
	\multicolumn{9}{l}{{Ridicule items}}  \\
	&Number of Mistakes & 1.80 & 1.00 & 2.14 & 0.00 & 16.00 & 2.95 & 14.73 & .72 \\ 
 &Participant would feel teasing & 0.93 & 0.00 & 1.99 & 0.00 & 16.00 & 4.73 & 28.70 & .84 \\ 
 &Goal of actor is teasing & 4.75 & 5.00 & 3.27 & 0.00 & 18.00 & 0.63 & 0.73 & .76 \\ 
 &Mean teasing component & 1.79 & 1.63 & 0.71 & 1.00 & 4.79 & 1.74 & 3.49 & .90 \\ 
 &Mean ridicule component & 5.37 & 5.47 & 0.48 & 2.74 & 6.00 & -2.09 & 7.64 & .82 \\ 
 &Mean funniness rating & 1.40 & 1.26 & 0.43 & 1.00 & 3.00 & 1.40 & 1.61 & .83 \\ 
 &Mean aversiveness rating & 5.19 & 5.37 & 0.60 & 3.58 & 6.00 & -0.67 & -0.33 & .86 \\ 

	\hline
\end{tabular}
\vspace{2mm}
\begin{tablenotes}
{\small\raggedright
\textit{Note.} 
N=122.
M$_{\textit{0}}$ = mean.
M$_{d}$ = median.
\textit{SD} = standard deviation.
Min = lowest value.
Max = highest value.
\textit{Sk} = skewness.
\textit{K} = kurtosis.
$\alpha$ = Cronbach's raw alpha. 
}
\end{tablenotes}
\end{center}
\end{table}

\begin{table}
	\begin{center}
	\caption{ Descriptives of PLaT Teasing and Ridicule Items for the Private Invitations Group}
	\begin{tabular}{llrrrrrrrr}
  	\hline
  	 & Variable & M$_{\textit{0}}$ & M$_{d}$ & \textit{SD} & Min & Max & \textit{Sk} & \textit{K} 	 &$\alpha$ \\
  	\hline 
		\multicolumn{9}{l}{{Teasing items}}  \\
		&Number of Mistakes & 0.47 & 0.00 & 0.87 & 0.00 & 3.00 & 1.67 & 1.75 & .49 \\ 
 &Participant would feel ridicule & 1.29 & 1.00 & 1.49 & 0.00 & 4.00 & 0.59 & -1.24 & .58 \\ 
 &Goal of actor is ridicule & 0.53 & 0.00 & 1.01 & 0.00 & 4.00 & 2.34 & 5.35 & .59 \\ 
 &Mean teasing component & 4.90 & 5.15 & 0.86 & 2.54 & 5.77 & -1.23 & 1.00 & .93 \\ 
 &Mean ridicule component & 1.69 & 1.69 & 0.52 & 1.00 & 3.15 & 1.16 & 1.19 & .85 \\ 
 &Mean funniness rating & 3.74 & 4.00 & 0.94 & 2.31 & 5.00 & -0.23 & -1.54 & .91 \\ 
 &Mean aversiveness rating & 1.62 & 1.46 & 0.47 & 1.08 & 2.54 & 0.66 & -0.84 & .69 \\ 
	\multicolumn{9}{l}{{Ridicule items}}  \\
	&Number of Mistakes & 1.41 & 1.00 & 2.00 & 0.00 & 8.00 & 2.10 & 4.15 & .73 \\ 
 &Participant would feel teasing & 0.76 & 0.00 & 1.48 & 0.00 & 6.00 & 2.56 & 6.27 & .70 \\ 
 &Goal of actor is teasing & 4.18 & 4.00 & 1.91 & 0.00 & 7.00 & -0.29 & -0.73 & .20 \\ 
 &Mean teasing component & 1.73 & 1.42 & 1.15 & 1.00 & 6.00 & 2.99 & 8.34 & .98 \\ 
 &Mean ridicule component & 5.23 & 5.47 & 0.63 & 3.53 & 5.95 & -1.20 & 0.70 & .88 \\ 
 &Mean funniness rating & 1.35 & 1.16 & 0.43 & 1.00 & 2.37 & 1.06 & -0.36 & .89 \\ 
 &Mean aversiveness rating & 5.18 & 5.26 & 0.66 & 3.68 & 6.00 & -0.40 & -0.70 & .89 \\ 

	\hline
\end{tabular}
\vspace{2mm}
\begin{tablenotes}
{\small\raggedright
\textit{Note.} 
N=17.
M$_{\textit{0}}$ = mean.
M$_{d}$ = median.
\textit{SD} = standard deviation.
Min = lowest value.
Max = highest value.
\textit{Sk} = skewness.
\textit{K} = kurtosis.
$\alpha$ = Cronbach's raw alpha. 
}
\end{tablenotes}
\end{center}
\end{table}




\begin{table}
	\begin{center}
	\caption{ Descriptives of PLaT Teasing and Ridicule Items for the Social Phobia Related Group}
	\begin{tabular}{llrrrrrrrr}
  	\hline
  	 & Variable & M$_{\textit{0}}$ & M$_{d}$ & \textit{SD} & Min & Max & \textit{Sk} & \textit{K} 	 &$\alpha$ \\
  	\hline 
		\multicolumn{9}{l}{{Teasing items}}  \\
		&Number of Mistakes & 0.71 & 1.00 & 0.85 & 0.00 & 3.00 & 1.23 & 1.07 & .22 \\ 
&Participant would feel ridicule & 1.75 & 2.00 & 1.35 & 0.00 & 4.00 & 0.09 & -1.30 & .28 \\ 
 &Goal of actor is ridicule & 0.64 & 0.00 & 0.99 & 0.00 & 4.00 & 1.83 & 3.12 & .43 \\ 
 &Mean teasing component & 4.90 & 4.81 & 0.63 & 3.69 & 5.92 & -0.13 & -1.00 & .85 \\ 
 &Mean ridicule component & 1.63 & 1.50 & 0.56 & 1.08 & 3.85 & 2.21 & 5.90 & .87 \\ 
 &Mean funniness rating & 3.27 & 3.31 & 0.98 & 1.62 & 5.31 & 0.38 & -0.78 & .91 \\ 
 &Mean aversiveness rating & 1.55 & 1.38 & 0.46 & 1.00 & 2.77 & 0.95 & 0.12 & .74 \\ 
	\multicolumn{9}{l}{{Ridicule items}}  \\
	&Number of Mistakes & 1.43 & 1.00 & 1.91 & 0.00 & 7.00 & 1.42 & 1.15 & .69 \\ 
 &Participant would feel teasing & 0.68 & 0.00 & 1.54 & 0.00 & 7.00 & 2.69 & 7.55 & .76 \\ 
 &Goal of actor is teasing & 3.64 & 3.00 & 3.15 & 0.00 & 12.00 & 0.74 & -0.16 & .76 \\ 
 &Mean teasing component & 1.51 & 1.45 & 0.42 & 1.00 & 2.53 & 0.85 & -0.11 & .80 \\ 
 &Mean ridicule component & 5.21 & 5.29 & 0.66 & 3.00 & 5.95 & -1.41 & 2.25 & .90 \\ 
 &Mean funniness rating & 1.31 & 1.13 & 0.42 & 1.00 & 2.63 & 1.57 & 1.75 & .85 \\ 
 &Mean aversiveness rating & 5.02 & 5.26 & 0.98 & 2.11 & 6.00 & -1.48 & 1.59 & .94 \\ 
	\hline
\end{tabular}
\vspace{2mm}
\begin{tablenotes}
{\small\raggedright
\textit{Note.} 
N=28.
M$_{\textit{0}}$ = mean.
M$_{d}$ = median.
\textit{SD} = standard deviation.
Min = lowest value.
Max = highest value.
\textit{Sk} = skewness.
\textit{K} = kurtosis.
$\alpha$ = Cronbach's raw alpha. 
}
\end{tablenotes}
\end{center}
\end{table}





\begin{table}
	\begin{center}
	\caption{ Descriptives of PLaT Teasing and Ridicule Items for the Autism Related Group}
	\begin{tabular}{llrrrrrrrr}
  	\hline
  	 & Variable & M$_{\textit{0}}$ & M$_{d}$ & \textit{SD} & Min & Max & \textit{Sk} & \textit{K} 	 &$\alpha$ \\
  	\hline 
		\multicolumn{9}{l}{{Teasing items}}  \\
		&Number of Mistakes & 1.13 & 0.00 & 1.74 & 0.00 & 7.00 & 1.75 & 2.60 & .72 \\ 
 &Participant would feel ridicule & 2.23 & 1.50 & 2.50 & 0.00 & 10.00 & 1.64 & 2.28 & .77 \\ 
 &Goal of actor is ridicule & 1.17 & 1.00 & 1.66 & 0.00 & 7.00 & 1.88 & 3.40 & .68 \\ 
 &Mean teasing component & 4.43 & 4.77 & 1.04 & 2.00 & 6.00 & -0.68 & -0.65 & .92 \\ 
 &Mean ridicule component & 1.88 & 1.77 & 0.65 & 1.00 & 3.31 & 0.72 & -0.61 & .85 \\ 
 &Mean funniness rating & 2.82 & 2.69 & 1.09 & 1.23 & 5.77 & 0.60 & -0.20 & .92 \\ 
 &Mean aversiveness rating & 1.93 & 1.65 & 0.89 & 1.00 & 4.77 & 1.37 & 1.49 & .91 \\ 
	\multicolumn{9}{l}{{Ridicule items}}  \\
	&Number of Mistakes & 1.20 & 1.00 & 1.61 & 0.00 & 5.00 & 1.38 & 0.63 & .61 \\ 
 &Participant would feel teasing & 0.60 & 0.00 & 1.22 & 0.00 & 5.00 & 2.31 & 4.80 & .65 \\ 
 &Goal of actor is teasing & 2.73 & 1.00 & 3.24 & 0.00 & 10.00 & 0.93 & -0.57 & .83 \\ 
 &Mean teasing component & 1.38 & 1.26 & 0.37 & 1.00 & 2.32 & 1.15 & 0.15 & .81 \\ 
 &Mean ridicule component & 5.45 & 5.58 & 0.49 & 3.95 & 6.00 & -1.13 & 0.87 & .83 \\ 
 &Mean funniness rating & 1.21 & 1.08 & 0.31 & 1.00 & 2.37 & 1.99 & 4.10 & .81 \\ 
 &Mean aversiveness rating & 5.14 & 5.50 & 0.99 & 1.21 & 6.00 & -2.24 & 5.80 & .94 \\ 
	\hline
\end{tabular}
\vspace{2mm}
\begin{tablenotes}
{\small\raggedright
\textit{Note.} 
N=30.
M$_{\textit{0}}$ = mean.
M$_{d}$ = median.
\textit{SD} = standard deviation.
Min = lowest value.
Max = highest value.
\textit{Sk} = skewness.
\textit{K} = kurtosis.
$\alpha$ = Cronbach's raw alpha. 
}
\end{tablenotes}
\end{center}
\end{table}




\begin{table}
	\begin{center}
	\caption{ Descriptives of RTSq Scenario Types for the Internet Forums Related Group}
	\begin{tabular}{llrrrrrrrr}
  	\hline
  	&Variable & M$_{\textit{0}}$ & M$_{d}$ & \textit{SD} & Min & Max & \textit{Sk} & \textit{K} &$\alpha$ \\
  	\hline 
\multicolumn{9}{l}{{Teasing scenarios}}  \\	
&Joy & 3.60 & 3.50 & 1.80 & 1.00 & 7.75 & 0.25 & -0.92 & .78 \\ 
 &Sadness & 2.91 & 2.50 & 1.88 & 1.00 & 8.00 & 0.89 & -0.13 & .87 \\ 
 &Anger & 3.43 & 3.00 & 1.82 & 1.00 & 8.00 & 0.56 & -0.69 & .80 \\ 
 &Disgust & 2.53 & 2.00 & 1.70 & 1.00 & 7.00 & 0.84 & -0.49 & .80 \\ 
 &Surprise & 4.20 & 4.50 & 1.49 & 1.00 & 8.00 & -0.07 & -0.26 & .56 \\ 
 &Shame & 3.51 & 3.00 & 2.02 & 1.00 & 8.00 & 0.55 & -0.64 & .84 \\ 
 &Fear & 2.93 & 2.00 & 2.15 & 1.00 & 8.00 & 0.95 & -0.43 & .87 \\ 
\multicolumn{9}{l}{{Ridicule scenarios}}  \\	
 &Joy & 2.16 & 1.75 & 1.29 & 1.00 & 6.25 & 1.04 & 0.33 & .66 \\ 
 &Sadness & 4.52 & 4.50 & 1.91 & 1.00 & 8.00 & 0.09 & -0.94 & .78 \\ 
 &Anger & 5.09 & 4.75 & 1.76 & 1.00 & 8.00 & -0.07 & -0.93 & .73 \\ 
 &Disgust & 3.47 & 3.25 & 1.85 & 1.00 & 8.00 & 0.47 & -0.76 & .76 \\ 
 &Surprise & 3.86 & 4.00 & 1.48 & 1.00 & 8.00 & 0.06 & -0.19 & .53 \\ 
 &Shame & 4.21 & 4.00 & 1.95 & 1.00 & 8.00 & 0.29 & -0.63 & .79 \\ 
 &Fear & 4.06 & 3.75 & 2.21 & 1.00 & 8.00 & 0.34 & -1.03 & .85 \\ 
\multicolumn{9}{l}{{Scenarios without social interaction}}  \\	
&Joy & 5.62 & 6.00 & 1.71 & 1.80 & 8.00 & -0.49 & -0.75 & .86 \\ 
 &Sadness & 1.65 & 1.20 & 0.94 & 1.00 & 6.00 & 2.15 & 5.74 & .76 \\ 
 &Anger & 1.58 & 1.20 & 0.80 & 1.00 & 3.80 & 1.33 & 0.58 & .62 \\ 
 &Disgust & 1.29 & 1.00 & 0.65 & 1.00 & 4.40 & 2.73 & 7.69 & .70 \\ 
 &Surprise & 2.41 & 2.20 & 1.24 & 1.00 & 7.00 & 0.98 & 0.89 & .75 \\ 
 &Shame & 1.50 & 1.00 & 0.87 & 1.00 & 4.60 & 1.95 & 2.90 & .70 \\ 
 &Fear & 1.57 & 1.00 & 1.08 & 1.00 & 6.00 & 2.21 & 4.41 & .78 \\ 
\hline
\end{tabular}
\vspace{2mm}
\begin{tablenotes}
{\small\raggedright
\textit{Note.} N=87.$\alpha$ = Cronbach's standardized alpha. \hfill \strut\\
}
\end{tablenotes}
\end{center}
\end{table}

\clearpage

\begin{table}
	\begin{center}
	\caption{ Descriptives of RTSq Scenario Types for the Psychology Students Related Group}
	\begin{tabular}{llrrrrrrrr}
  	\hline
  	&Variable & M$_{\textit{0}}$ & M$_{d}$ & \textit{SD} & Min & Max & \textit{Sk} & \textit{K} &$\alpha$ \\
  	\hline 
\multicolumn{9}{l}{{Teasing scenarios}}  \\	
&Joy & 3.84 & 3.50 & 1.67 & 1.00 & 7.50 & 0.20 & -0.94 & .78 \\ 
 &Sadness & 2.70 & 2.50 & 1.35 & 1.00 & 6.50 & 0.55 & -0.62 & .66 \\ 
 &Anger & 3.14 & 3.00 & 1.48 & 1.00 & 7.00 & 0.31 & -0.86 & .69 \\ 
 &Disgust & 2.59 & 2.25 & 1.47 & 1.00 & 7.00 & 0.81 & -0.08 & .69 \\ 
 &Surprise & 3.99 & 4.00 & 1.52 & 1.00 & 8.00 & 0.00 & -0.61 & .66 \\ 
 &Shame & 3.28 & 3.25 & 1.61 & 1.00 & 7.50 & 0.34 & -0.87 & .73 \\ 
 &Fear & 2.49 & 2.12 & 1.60 & 1.00 & 8.00 & 1.40 & 1.88 & .82 \\ 

\multicolumn{9}{l}{{Ridicule scenarios}}  \\	
	&Joy & 1.94 & 1.75 & 0.95 & 1.00 & 6.25 & 1.22 & 2.06 & .51 \\ 
 &Sadness & 4.63 & 4.75 & 1.57 & 1.00 & 8.00 & -0.29 & -0.67 & .69 \\ 
 &Anger & 5.07 & 5.25 & 1.61 & 1.00 & 8.00 & -0.28 & -0.56 & .75 \\ 
 &Disgust & 4.07 & 4.00 & 1.68 & 1.00 & 7.50 & 0.05 & -0.88 & .68 \\ 
 &Surprise & 4.19 & 4.25 & 1.55 & 1.00 & 7.75 & -0.04 & -0.59 & .67 \\ 
 &Shame & 4.54 & 4.25 & 1.72 & 1.00 & 8.00 & -0.06 & -0.75 & .74 \\ 
 &Fear & 4.19 & 4.25 & 1.67 & 1.00 & 8.00 & 0.06 & -0.52 & .74 \\ 

\multicolumn{9}{l}{{Scenarios without social interaction}}  \\	
	&Joy & 6.39 & 6.40 & 1.02 & 2.80 & 8.00 & -0.65 & 0.23 & .86 \\ 
 &Sadness & 1.23 & 1.00 & 0.47 & 1.00 & 4.40 & 3.47 & 17.20 & .76 \\ 
 &Anger & 1.45 & 1.40 & 0.53 & 1.00 & 3.00 & 1.23 & 0.80 & .62 \\ 
&Disgust & 1.11 & 1.00 & 0.33 & 1.00 & 3.20 & 3.99 & 18.07 & .70 \\ 
&Surprise & 2.44 & 2.20 & 1.24 & 1.00 & 6.40 & 0.95 & 0.30 & .75 \\ 
 &Shame & 1.25 & 1.00 & 0.46 & 1.00 & 3.20 & 2.42 & 5.68 & .70 \\ 
 &Fear & 1.22 & 1.00 & 0.47 & 1.00 & 4.40 & 3.60 & 17.29 & .78 \\ 

\hline
\end{tabular}
\vspace{2mm}
\begin{tablenotes}
{\small\raggedright
\textit{Note.} 
N=122.
$\alpha$ = Cronbach's standardized alpha. \hfill \strut\\
}
\end{tablenotes}
\end{center}
\end{table}



\begin{table}
	\begin{center}
	\caption{ Descriptives of RTSq Scenario Types for the Private Invitations Group}
	\begin{tabular}{llrrrrrrrr}
  	\hline
  	&Variable & M$_{\textit{0}}$ & M$_{d}$ & \textit{SD} & Min & Max & \textit{Sk} & \textit{K} &$\alpha$ \\
  	\hline 
\multicolumn{9}{l}{{Teasing scenarios}}  \\	
&Joy & 3.72 & 3.50 & 2.05 & 1.00 & 7.25 & 0.17 & -1.44 & .88 \\ 
 &Sadness & 2.79 & 2.75 & 1.38 & 1.00 & 4.75 & 0.03 & -1.81 & .83 \\ 
 &Anger & 3.04 & 3.25 & 1.44 & 1.00 & 5.75 & 0.29 & -1.12 & .75 \\ 
 &Disgust & 2.51 & 2.25 & 1.34 & 1.00 & 5.50 & 0.63 & -0.71 & .76 \\ 
 &Surprise & 4.46 & 4.75 & 1.68 & 1.00 & 7.25 & -0.42 & -0.74 & .77 \\ 
 &Shame & 3.28 & 2.75 & 1.81 & 1.25 & 7.25 & 0.57 & -0.90 & .88 \\ 
 &Fear & 2.29 & 2.00 & 1.26 & 1.00 & 5.00 & 0.65 & -0.92 & .80 \\ 
\multicolumn{9}{l}{{Ridicule scenarios}}  \\	
&Joy & 1.84 & 1.50 & 1.14 & 1.00 & 5.00 & 1.47 & 1.20 & .81 \\ 
 &Sadness & 4.62 & 4.75 & 1.34 & 2.50 & 6.75 & -0.18 & -1.30 & .53 \\ 
 &Anger & 4.76 & 5.25 & 1.68 & 2.00 & 7.50 & -0.03 & -1.19 & .73 \\ 
 &Disgust & 3.71 & 3.25 & 1.93 & 1.00 & 7.00 & 0.26 & -1.31 & .79 \\ 
 &Surprise & 4.10 & 4.25 & 1.43 & 1.75 & 7.00 & 0.39 & -0.63 & .54 \\ 
 &Shame & 4.60 & 4.75 & 2.04 & 1.25 & 8.00 & -0.07 & -1.33 & .82 \\ 
 &Fear & 4.81 & 4.75 & 1.96 & 1.50 & 7.75 & 0.04 & -1.43 & .78 \\ 
\multicolumn{9}{l}{{Scenarios without social interaction}}  \\	
&Joy & 5.73 & 6.20 & 1.66 & 2.80 & 8.00 & -0.31 & -1.44 & .86 \\ 
 &Sadness & 1.46 & 1.00 & 1.00 & 1.00 & 5.00 & 2.63 & 6.44 & .76 \\ 
 &Anger & 1.36 & 1.20 & 0.53 & 1.00 & 2.80 & 1.45 & 1.01 & .62 \\ 
 &Disgust & 1.15 & 1.00 & 0.25 & 1.00 & 1.80 & 1.32 & 0.43 & .70 \\ 
 &Surprise & 2.44 & 2.00 & 1.30 & 1.00 & 5.00 & 0.97 & -0.50 & .75 \\ 
 &Shame & 1.52 & 1.00 & 0.92 & 1.00 & 4.00 & 1.74 & 1.64 & .70 \\ 
 &Fear & 1.35 & 1.00 & 0.83 & 1.00 & 4.40 & 2.88 & 7.75 & .78 \\ 
\hline
\end{tabular}
\vspace{2mm}
\begin{tablenotes}
{\small\raggedright
\textit{Note.} 
N=17.
$\alpha$ = Cronbach's standardized alpha.\hfill \strut\\
}
\end{tablenotes}
\end{center}
\end{table}


\begin{table}
	\begin{center}
	\caption{ Descriptives of RTSq Scenario Types for the Social Phobia Related Group}
	\begin{tabular}{llrrrrrrrr}
  	\hline
  	&Variable & M$_{\textit{0}}$ & M$_{d}$ & \textit{SD} & Min & Max & \textit{Sk} & \textit{K} &$\alpha$ \\
  	\hline 
\multicolumn{9}{l}{{Teasing scenarios}}  \\	
&Joy & 2.97 & 3.00 & 1.25 & 1.00 & 5.50 & 0.17 & -0.69 & .53 \\ 
 &Sadness & 3.23 & 3.12 & 1.49 & 1.00 & 6.50 & 0.46 & -0.55 & .69 \\ 
 &Anger & 3.33 & 3.38 & 1.70 & 1.00 & 6.75 & 0.15 & -1.19 & .70 \\ 
 &Disgust & 2.62 & 2.25 & 1.61 & 1.00 & 6.50 & 0.91 & -0.30 & .77 \\ 
 &Surprise & 3.89 & 3.75 & 1.31 & 1.75 & 6.75 & 0.30 & -0.99 & .46 \\ 
 &Shame & 4.79 & 4.75 & 1.88 & 1.00 & 8.00 & -0.01 & -1.02 & .75 \\ 
 &Fear & 4.27 & 4.50 & 1.85 & 1.00 & 8.00 & -0.03 & -0.85 & .73 \\ 

\multicolumn{9}{l}{{Ridicule scenarios}}  \\	
	&Joy & 1.61 & 1.38 & 0.93 & 1.00 & 4.75 & 2.14 & 4.11 & .60 \\ 
 &Sadness & 4.62 & 4.75 & 1.61 & 1.00 & 7.75 & -0.36 & -0.46 & .72 \\ 
 &Anger & 4.92 & 5.12 & 1.68 & 1.50 & 7.50 & -0.43 & -0.69 & .72 \\ 
 &Disgust & 3.82 & 3.25 & 1.99 & 1.00 & 7.50 & 0.23 & -1.33 & .79 \\ 
 &Surprise & 3.45 & 3.62 & 1.71 & 1.00 & 6.75 & 0.36 & -1.12 & .81 \\ 
 &Shame & 5.34 & 5.38 & 1.98 & 1.25 & 8.00 & -0.42 & -0.92 & .84 \\ 
 &Fear & 5.34 & 5.12 & 2.09 & 1.25 & 8.00 & -0.34 & -1.14 & .83 \\ 

\multicolumn{9}{l}{{Scenarios without social interaction}}  \\	
&Joy & 5.31 & 5.30 & 1.80 & 1.20 & 7.80 & -0.51 & -0.82 & .86 \\ 
 &Sadness & 1.66 & 1.40 & 0.73 & 1.00 & 4.00 & 1.34 & 1.56 & .76 \\ 
 &Anger & 1.43 & 1.20 & 0.53 & 1.00 & 2.80 & 1.28 & 0.43 & .62 \\ 
 &Disgust & 1.22 & 1.00 & 0.37 & 1.00 & 2.20 & 1.60 & 1.33 & .70 \\ 
 &Surprise & 2.46 & 2.00 & 1.31 & 1.00 & 6.20 & 1.06 & 0.35 & .75 \\ 
 &Shame & 1.69 & 1.60 & 0.69 & 1.00 & 4.00 & 1.29 & 2.04 & .70 \\ 
 &Fear & 1.94 & 1.80 & 0.83 & 1.00 & 4.20 & 0.84 & 0.02 & .78 \\ 

\hline
\end{tabular}
\vspace{2mm}
\begin{tablenotes}
{\small\raggedright
\textit{Note.} 
N=28.
$\alpha$ = Cronbach's standardized alpha. \hfill \strut\\
}
\end{tablenotes}
\end{center}
\end{table}


\begin{table}
	\begin{center}
	\caption{ Descriptives of RTSq Scenario Types for the Autism Related Group}
	\begin{tabular}{llrrrrrrrr}
  	\hline
  	&Variable & M$_{\textit{0}}$ & M$_{d}$ & \textit{SD} & Min & Max & \textit{Sk} & \textit{K} &$\alpha$ \\
  	\hline 
\multicolumn{9}{l}{{Teasing scenarios}}  \\	
&Joy & 2.06 & 1.62 & 1.28 & 1.00 & 6.00 & 1.47 & 1.72 & .78 \\ 
 &Sadness & 3.52 & 3.25 & 2.07 & 1.00 & 7.25 & 0.33 & -1.25 & .82 \\ 
 &Anger & 4.08 & 4.12 & 1.83 & 1.00 & 7.25 & -0.02 & -0.84 & .75 \\ 
 &Disgust & 3.42 & 3.50 & 1.95 & 1.00 & 7.00 & 0.16 & -1.30 & .77 \\ 
 &Surprise & 4.11 & 4.25 & 1.83 & 1.00 & 7.50 & 0.07 & -1.20 & .67 \\ 
 &Shame & 4.57 & 4.75 & 2.15 & 1.00 & 7.50 & -0.25 & -1.42 & .79 \\ 
 &Fear & 3.94 & 3.62 & 2.29 & 1.00 & 7.75 & 0.18 & -1.39 & .86 \\ 

\multicolumn{9}{l}{{Ridicule scenarios}}  \\	
	&Joy & 1.32 & 1.00 & 0.67 & 1.00 & 4.00 & 2.45 & 6.18 & .53 \\ 
 &Sadness & 4.24 & 4.50 & 2.13 & 1.00 & 8.00 & -0.11 & -1.26 & .85 \\ 
 &Anger & 4.88 & 5.25 & 1.92 & 1.75 & 8.00 & 0.06 & -1.30 & .77 \\ 
 &Disgust & 3.89 & 3.88 & 2.10 & 1.00 & 7.50 & 0.01 & -1.45 & .85 \\ 
 &Surprise & 3.77 & 4.00 & 1.76 & 1.00 & 7.00 & 0.05 & -1.17 & .64 \\ 
 &Shame & 4.66 & 5.38 & 2.25 & 1.00 & 8.00 & -0.40 & -1.25 & .87 \\ 
 &Fear & 4.80 & 5.12 & 2.34 & 1.00 & 8.00 & -0.32 & -1.37 & .87 \\ 

\multicolumn{9}{l}{{Scenarios without social interaction}}  \\	
&Joy & 3.50 & 3.60 & 1.67 & 1.00 & 8.00 & 0.67 & 0.24 & .86 \\ 
 &Sadness & 1.57 & 1.20 & 0.88 & 1.00 & 4.20 & 1.66 & 1.59 & .76 \\ 
 &Anger & 2.13 & 1.90 & 1.02 & 1.00 & 4.20 & 0.51 & -1.11 & .62 \\ 
 &Disgust & 1.43 & 1.00 & 0.68 & 1.00 & 4.00 & 1.93 & 4.17 & .70 \\ 
 &Surprise & 2.49 & 2.00 & 1.36 & 1.00 & 5.40 & 0.71 & -0.72 & .75 \\ 
 &Shame & 1.66 & 1.30 & 0.92 & 1.00 & 4.20 & 1.43 & 0.94 & .70 \\ 
 &Fear & 1.58 & 1.20 & 0.80 & 1.00 & 4.00 & 1.51 & 1.53 & .78 \\ 

\hline
\end{tabular}
\vspace{2mm}
\begin{tablenotes}
{\small\raggedright
\textit{Note.} 
N=30.
$\alpha$ = Cronbach's standardized alpha.\hfill \strut\\
}
\end{tablenotes}
\end{center}
\end{table}



%%%%%%%%%%%%%%%%%%%%%%%%%%%

\begin{table}
	\begin{center}
	\caption{Descriptives of Frequencies of Being Mobbed and Emotional Affect for the Internet Forums Related Group}
	\begin{tabular}{lrrrrrrr}
  	\hline
  	Variable & M$_{\textit{0}}$ & M$_{d}$ & \textit{SD} & Min & Max & \textit{Sk} & \textit{K} \\
  	\hline 
	Physically & 1.80 & 2.00 & 0.87 & 1.00 & 4.00 & 0.90 & 0.05 \\ 
 Properties & 1.63 & 1.00 & 0.79 & 1.00 & 4.00 & 1.01 & 0.16 \\ 
 Verbally & 2.43 & 2.00 & 0.97 & 1.00 & 4.00 & 0.21 & -0.97 \\ 
 Socially & 2.36 & 2.00 & 1.14 & 1.00 & 4.00 & 0.17 & -1.41 \\ 
 Ridiculed & 2.47 & 2.00 & 1.09 & 1.00 & 4.00 & 0.18 & -1.30 \\ 
 Emotional affect & 2.36 & 2.00 & 0.75 & 1.00 & 4.00 & -0.34 & -0.69 \\ 
	\hline
\end{tabular}
\vspace{2mm}
\begin{tablenotes}
{\small\raggedright
\textit{Note.} 
N=87.
M$_{\textit{0}}$ = mean.
M$_{d}$ = median.
\textit{SD} = standard deviation.
Min = lowest value.
Max = highest value.
\textit{Sk} = skewness.
\textit{K} = kurtosis.
Emotional affect is negatively poled.\\
}
\end{tablenotes}
\end{center}
\end{table}


\begin{table}
	\begin{center}
	\caption{Descriptives of Frequencies of Being Mobbed and Emotional Affect for the Psychology Students Related Group}
	\begin{tabular}{lrrrrrrr}
  	\hline
  	Variable & M$_{\textit{0}}$ & M$_{d}$ & \textit{SD} & Min & Max & \textit{Sk} & \textit{K} \\
  	\hline 
	Physically & 1.40 & 1.00 & 0.75 & 1.00 & 4.00 & 1.82 & 2.48 \\ 
 Properties & 1.38 & 1.00 & 0.71 & 1.00 & 4.00 & 1.82 & 2.51 \\ 
 Verbally & 2.23 & 2.00 & 0.87 & 1.00 & 4.00 & 0.44 & -0.42 \\ 
 Socially & 2.19 & 2.00 & 0.93 & 1.00 & 4.00 & 0.42 & -0.68 \\ 
 Ridiculed & 2.11 & 2.00 & 0.90 & 1.00 & 4.00 & 0.45 & -0.60 \\ 
 Emotional affect & 1.98 & 2.00 & 0.55 & 1.00 & 4.00 & 0.29 & 1.48 \\ 
	\hline
\end{tabular}
\vspace{2mm}
\begin{tablenotes}
{\small\raggedright
\textit{Note.} 
N=122.
M$_{\textit{0}}$ = mean.
M$_{d}$ = median.
\textit{SD} = standard deviation.
Min = lowest value.
Max = highest value.
\textit{Sk} = skewness.
\textit{K} = kurtosis.
Emotional affect is negatively poled.\\
}
\end{tablenotes}
\end{center}
\end{table}

\begin{table}
	\begin{center}
	\caption{Descriptives of Frequencies of Being Mobbed and Emotional Affect for the Private Invitations Group}
	\begin{tabular}{lrrrrrrr}
  	\hline
  	Variable & M$_{\textit{0}}$ & M$_{d}$ & \textit{SD} & Min & Max & \textit{Sk} & \textit{K} \\
  	\hline 
	Physically & 1.53 & 1.00 & 0.62 & 1.00 & 3.00 & 0.62 & -0.78 \\ 
 Properties & 1.59 & 2.00 & 0.62 & 1.00 & 3.00 & 0.43 & -0.92 \\ 
 Verbally & 2.35 & 2.00 & 0.93 & 1.00 & 4.00 & 0.19 & -0.99 \\ 
 Socially & 2.59 & 2.00 & 1.00 & 1.00 & 4.00 & 0.12 & -1.29 \\ 
 Ridiculed & 2.29 & 2.00 & 0.92 & 1.00 & 4.00 & 0.35 & -0.83 \\ 
 Emotional affect & 2.12 & 2.00 & 0.70 & 1.00 & 3.00 & -0.13 & -1.07 \\ 
	\hline
\end{tabular}
\vspace{2mm}
\begin{tablenotes}
{\small\raggedright
\textit{Note.} 
N=17.
M$_{\textit{0}}$ = mean.
M$_{d}$ = median.
\textit{SD} = standard deviation.
Min = lowest value.
Max = highest value.
\textit{Sk} = skewness.
\textit{K} = kurtosis.\\
}
\end{tablenotes}
\end{center}
\end{table}


\begin{table}
	\begin{center}
	\caption{Descriptives of Frequencies of Being Mobbed and Emotional Affect for the Social Phobia Related Group}
	\begin{tabular}{lrrrrrrr}
  	\hline
  	Variable & M$_{\textit{0}}$ & M$_{d}$ & \textit{SD} & Min & Max & \textit{Sk} & \textit{K} \\
  	\hline 
	Physically & 1.54 & 1.00 & 0.69 & 1.00 & 3.00 & 0.84 & -0.59 \\ 
 Properties & 1.57 & 1.00 & 0.74 & 1.00 & 3.00 & 0.81 & -0.80 \\ 
 Verbally & 2.57 & 3.00 & 1.03 & 1.00 & 4.00 & -0.09 & -1.23 \\ 
 Socially & 2.82 & 3.00 & 0.86 & 1.00 & 4.00 & 0.00 & -1.13 \\ 
 Ridiculed & 2.96 & 3.00 & 0.88 & 1.00 & 4.00 & -0.25 & -1.09 \\ 
 Emotional affect & 2.50 & 2.50 & 0.75 & 1.00 & 4.00 & 0.00 & -0.48 \\ 
	\hline
\end{tabular}
\vspace{2mm}
\begin{tablenotes}
{\small\raggedright
\textit{Note.} 
N=28.
M$_{\textit{0}}$ = mean.
M$_{d}$ = median.
\textit{SD} = standard deviation.
Min = lowest value.
Max = highest value.
\textit{Sk} = skewness.
\textit{K} = kurtosis.
Emotional affect is negatively poled.\\
}
\end{tablenotes}
\end{center}
\end{table}

\begin{table}
	\begin{center}
	\caption{Descriptives of Frequencies of Being Mobbed and Emotional Affect for the Autism Related Group}
	\begin{tabular}{lrrrrrrr}
  	\hline
  	Variable & M$_{\textit{0}}$ & M$_{d}$ & \textit{SD} & Min & Max & \textit{Sk} & \textit{K} \\
  	\hline 
	Physically & 2.30 & 2.00 & 0.95 & 1.00 & 4.00 & 0.10 & -1.07 \\ 
 Properties & 2.30 & 2.00 & 1.06 & 1.00 & 4.00 & 0.26 & -1.22 \\ 
 Verbally & 3.33 & 3.00 & 0.71 & 2.00 & 4.00 & -0.54 & -0.97 \\ 
 Socially & 3.70 & 4.00 & 0.53 & 2.00 & 4.00 & -1.46 & 1.16 \\ 
 Ridiculed & 3.37 & 4.00 & 0.85 & 1.00 & 4.00 & -1.06 & 0.07 \\ 
 Emotional affect & 3.00 & 3.00 & 0.64 & 1.00 & 4.00 & -0.75 & 1.67 \\ 
	\hline
\end{tabular}
\vspace{2mm}
\begin{tablenotes}
{\small\raggedright
\textit{Note.} 
N=30.
M$_{\textit{0}}$ = mean.
M$_{d}$ = median.
\textit{SD} = standard deviation.
Min = lowest value.
Max = highest value.
\textit{Sk} = skewness.
\textit{K} = kurtosis.
Emotional affect is negatively poled.\\
}
\end{tablenotes}
\end{center}
\end{table}

\clearpage
 
\section{PLaT Items}
\label{platitems}
\addcontentsline{toc}{section}{Appendix~\ref{platitems}: PlaT Items}
   
Teasing items:
   
T1. Rafael ist bei seinen Mitbewohnern als Computerfreak bekannt, der seine Zeit meist alleine vor dem Computer verbringt. Einmal kommt ein Mitbewohner von der Arbeit nach Hause und erz\"ahlt, wie er heute den ganzen Tag alleine im B\"uro vor dem PC gesessen habe und mit keinem Menschen sprechen konnte. Da beginnt der Mitbewohner pl\"otzlich zu lachen: ``Es kam mir vor, als ob ich du w\"are.''


T2. Christoph hat heute den ganzen Sonntag die Fenster seines Einfamilienhauses geputzt und ist nun gerade daran, sein Putzzeug wegzur\"aumen. Sein Nachbar, der ihn dabei beobachtet, macht sich einen Spass daraus, Christoph Stellen zu zeigen, die noch nicht ganz sauber sind: ``Dort sehe ich noch ein Staubkorn!''


T3. Lea ist ein Morgenmuffel. Als Lea eines Morgens wieder einmal total verschlafen in die K\"uche kommt, lacht ihre Mitbewohnerin sie an und meint: ``Du schaust aus als h\"attest du eine sehr anstrengende Nacht hinter dir. Wie viele Gegner hast du im Kampf besiegt?''


T4. Nico ist bei seinen Mitbewohnern bekannt daf\"ur, dass er an Partys reihenweise Frauen abschleppt. Beim Abendessen erz\"ahlt Nico, dass er vorhabe nach Amerika auszuwandern. Darauf beginnt sein Mitbewohner zu lachen: ``Hast du denn die Schweizer Frauen schon alle durch?''
 
 
 T5. Bruno ist oft zerstreut und kommt, nachdem er aus dem Haus gegangen ist, meistens nochmals zur\"uck, weil er etwas vergessen hat. Als Bruno eines Morgens das Haus verl\"asst, ruft ihm sein Mitbewohner lachend nach: ``Tschau, bis in einer Minute.''


T6. Alex ist ein Besserwisser. Allerdings ist er sich dessen bewusst und kann dar\"uber lachen, wenn jemand diesbez\"uglich einen Spruch macht. Als am Arbeitsplatz \"uber die Wirtschaftskrise diskutiert wird, hat er f\"ur alles eine L\"osung parat. Darauf meint ein Mitarbeiter lachend: ``Schade, dass du nicht Pr\"asident der Welt bist. Dann w\"urde es uns allen besser gehen.''


T7. Beatrice wartet auf eine Bekannte. Als diese ankommt, schleicht sie sich von links an Beatrice heran und klopft ihr auf die rechte Schulter. Beatrice dreht sich nach rechts um und fragt sich ganz verdutzt, wer ihr denn nun auf die Schulter geklopft habe. Die Bekannte freut sich \"uber den gelungenen Spass und lacht lauthals.
 
 
 T8. Rahel spaziert mit ihrer besten Freundin an dem Jungen vorbei, in den sie heimlich verliebt ist. Sie err\"otet und wird nerv\"os. Ihre Freundin l\"achelt sie darauf an und formt mit den H\"anden ein Herzchen.
 
 
T9. Oft versch\"uttet Markus sein Getr\"ank auf seine Kleider oder auf den Tisch. Seine Freundin muss jedes Mal lachen, wenn dies passiert. Als Markus wieder einmal sein Bier \"uber die Hosen kippt, nennt sie ihn mit einem lieben L\"acheln ``Tollpatsch''.
  
  
T10. Jan spielt \"ofters mit einem Mitarbeiter ein Brettspiel in der Mittagspause. Dieser Mitarbeiter hat heute, im Unterschied zum letzten Mal, unglaubliches Gl\"uck. Er w\"urfelt fast immer genau die Zahl, die er braucht und gewinnt deshalb mehrmals hintereinander. Am Nachmittag spricht er Jan immer wieder lachend darauf an, wie es ihm nach den vielen Niederlagen denn so gehe und ob er vielleicht ein Glas Schnaps zur Verarbeitung der Niederlagen wolle.
  
  
  T11. Peter ist bei seinen Arbeitskollegen bekannt daf\"ur immer zu sp\"at zu Anl\"assen zu erscheinen. Als Peter f\"ur einmal p\"unktlich erscheint, meint ein Mitarbeiter lachend: ``Hey, was ist mit dir los? Du bist drei Sekunden zu fr\"uh!''
   
  
T12. Max hat bei seiner Verwandtschaft den Ruf auf dem Teller stets ein ``Schlachtfeld'' aus Sauce und Speiseresten zu hinterlassen. Beim Weihnachtsessen meint sein Schwager nach dem Hauptgang lachend: ``So muss es nach der Schlacht bei Waterloo ausgesehen haben.''
   
 
T13. In der Fahrschule hat Diego M\"uhe mit dem Anfahren am Hang. Er und sein Fahrlehrer kommen sehr gut miteinander aus und scherzen oft gemeinsam. Einmal, als Diego am Hang anfahren will, meint der Fahrlehrer l\"achelnd, Beat m\"oge kurz anhalten, damit er aussteigen und helfen k\"onne, den Wagen anzuschieben.
   subsection{Ridicule Items} 
  
  
Ridicule items:
  
R1. Lars ist 14 Jahre alt und hat schon seit er klein war grosse Freude an Modelleisenbahnen. Als er Mitsch\"ulern von seiner Faszination erz\"ahlt, beginnen diese laut zu lachen: ``Was? Du spielst immer noch mit Kinderspielzeug? Schl\"afst du auch noch mit deinem Lieblingsteddy?''
  
  
R2. Melina geht schon seit Jahren in den Kirchenchor, da sie es liebt, mit anderen Menschen zusammen zu singen. In der Chorprobe h\"ort Melina, wie vor ihr zwei Frauen leise dar\"uber kichern, wie Melina jeden zweiten Ton nicht treffe und ausserdem eine ganz schreckliche Stimme habe.
  
  
R3. Cornelia besucht seit langem wieder einmal ihre Tante und ihren Onkel. Ihre Tante stellt fest: ``Gesund siehst du aus!'' Darauf f\"allt ihr der Onkel mit einem h\"ohnischen Lachen ins Wort: ``Gesund sagst du dem? Die ist ja richtig fettleibig geworden. Schau sie dir doch einmal an, wenn die baden geht ist nachher kein Wasser mehr im Schwimmbecken.''
  
  
R4. Corinne ist alleinerziehende Mutter und ihr Einkommen reicht ihr nur knapp. Deshalb muss sie an vielen Orten sparen und kann sich keine teuren Kleider leisten, ganz im Unterschied zu ihren beiden Arbeitskolleginnen. Diese sind beide finanziell abgesichert und erscheinen stets in teuren Markenkleidern. Als Corinne einmal in einem eleganten Deux-piece erscheint, lachen ihre Arbeitskolleginnen und meinen: ``Wo hast du denn dieses Kleid gestohlen? Ehrlich gekauft kannst du es wohl kaum haben.''
  
  
R5. Matthias ist sehr korpulent. Als er zusammen mit seinen Arbeitskollegen den Lift betreten will, meinen diese, er m\"usse draussen bleiben, da sonst das Maximalgewicht \"uberschritten werde. Als Matthias trotzdem in den Lift steigen will, versperren sie ihm demonstrativ den Weg. Darauf lachen sie dar\"uber, dass sie jetzt so gerade einen Unfall verhindert h\"atten und fahren ohne ihn los.
 
 
R6. Lorenz hat eine sehr grosse Nase, die er selbst gar nicht sch\"on findet. Aus diesem Grund mag er es auch nicht, wenn irgendjemand eine Bemerkung dar\"uber macht. An einem Gesch\"aftsessen begr\"usst ihn ein Arbeitskollege und meint lachend: ``Du hast da ein riesiges Geschw\"ur, ich kann dein Gesicht gar nicht sehen'' worauf alle Mitarbeiter zu lachen beginnen.
  
  
R7. Thomas hat M\"uhe sich franz\"osische W\"orter zu merken und bringt regelm\"assig Silben durcheinander, was seinen Franz\"osischlehrer immer wieder belustigt. Als er kontrollieren will, ob die Sch\"uler die aufgegebenen Vokabeln gelernt haben, ruft er regelm\"assig Thomas auf und grinst dabei h\"amisch. Als Thomas dann die Silben eines Wortes durcheinander bringt, beginnt der Lehrer laut zu lachen.
  
  
R8. Frieda wird an ihrem Arbeitsplatz immer wieder wegen ihres altmodischen Namens geh\"anselt. Als Frieda den Arbeitsraum betritt, ruft eine Mitarbeiterin laut : ``Hey Frieda, wie ist es so im Altersheim? Hast du dein neues Gebiss schon erhalten?'' Darauf erwidert Frieda gekr\"ankt, dass sie gar nicht in einem Altersheim wohne, worauf die Mitarbeiterin h\"amisch meint: ``Und Alzheimer scheinst du auch noch zu haben.''
  
  
R9. Kevin m\"ochte im Zeichenunterricht m\"oglichst ungest\"ort sein und sitzt daher alleine an eine Bank. Ein Mitsch\"uler von ihm setzt sich absichtlich neben Kevin, um ihn zu \"argern. Er zeichnet w\"ahrend des ganzen Unterrichts immer wieder Striche auf Kevins Zeichnung. Jedes Mal, wenn Kevin den Strich wieder ver\"argert wegradiert, lacht der Mitsch\"uler dar\"uber.
  
  
R10. Daniel stellt an seinem Arbeitsplatz erstaunt fest, dass an der Wand, an der ein Foto von jedem Mitarbeiter h\"angt, jemand sein Foto mit einem anderen \"uberklebt hat. Das Foto zeigt ihn auf der Toilette. Jemand muss das Foto heimlich erstellt haben.
 
 
R11. Petra sitzt in der Cafeteria zwei unbekannten gleichaltrigen M\"annern gegen\"uber. Diese diskutieren f\"ur Petra deutlich h\"orbar dar\"uber, wie seltsam ihre Nase aussehe. Einer von ihnen meint, vielleicht habe sie die Nase eines Schweines transplantiert bekommen, worauf beide lachen.
  
  
R12. Mario stottert. Einmal als Mario seinen Vorgesetzten etwas fragen will, gibt dieser keine Antwort, sondern \"afft nur sein Stottern nach. Als Mario seine Frage erneut stellt, meint sein Vorgesetzter lachend, er solle zuerst richtig sprechen lernen, bevor er eine Antwort erwarten d\"urfe.
  
  
R13. Sonja hat rote Haare, wor\"uber ihre Mitsch\"ulerinnen oft Spr\"uche machen, die Sonja gar nicht mag. Einmal, als Sonja zur Schule kommt, haben ein paar Mitsch\"ulerinnen bereits auf sie gewartet, um im Chor zu rufen: ``Rotkopf die Ecke brennt, Feuerwehr kommt angerannt.'' Darauf sch\"utten sie ihr einen Eimer kaltes Wasser \"uber den Kopf und lachen.
  
  
R14. J\"urg ist unsportlich und tr\"age. Als im Sportunterricht die Fussballmannschaften gebildet werden, wird J\"urg wieder einmal als Letzter gew\"ahlt. Darauf rufen einige Spieler der gegnerischen Mannschaft: ``Haha, ihr m\"usst J\"urg in der Mannschaft haben'' und jedes Mal wenn er im Spiel den Ball nicht richtig trifft beginnen sie zu lachen.
 
 
R15. Michael ist Afrikaner und mit einer Deutschen verheiratet. Er wird von deren Bruder manchmal absch\"atzig als ``Neger'' bezeichnet. Einmal an einem Familienessen fragt der Bruder Michael, ob er so braun sei, weil er ins Plumpsklo gefallen sei. Er sehe so verschissen aus. Darauf beginnt er laut zu lachen.
 
  
R16. Lukas ist ein Aussenseiter in seiner Schulklasse. Einmal im Kochunterricht beginnen ein paar seiner Mitsch\"uler mit dem Geschirrtuch auf ihn einzuschlagen. Als Lukas gekr\"ankt fragt, warum sie das machen, sagt einer von ihnen lachend, er sei eben noch nicht ganz trocken und schl\"agt weiter auf Lukas ein.
   
    
R17. Melanie hat nur wenige Freunde an ihrer Arbeitsstelle. Auf Facebook entdeckt sie, dass es dort eine Diskussionsgruppe mit dem Namen ``Melanie ist doof'' gibt. Dort machen Mitarbeiterinnen und Mitarbeiter Witze dar\"uber, weshalb Melanie wohl so h\"asslich sei und wie schrecklich es f\"ur sie sei, Melanie jeden Tag w\"ahrend der Arbeit anschauen zu m\"ussen.
   
 
R18. Nick ist verschlossen und \"uberhaupt nicht schlagfertig. Er wird \"ofters von seinen Mitsch\"ulern gemobbt. Eines Morgens kommt ein Mitsch\"uler in das Klassenzimmer und fragt einen anderen Mitsch\"uler laut: ``Wollen wir heute wieder Nick fertig machen?'' Darauf antwortet dieser Misch\"uler: ``Das ist doch bei einem so armseligen Menschen wie Nick gar nicht mehr n\"otig'', worauf beide lachen.
   
  
R19. Ir�ne ist f\"unf Jahre alt und mit ihren Eltern zu Besuch bei Bekannten, die zwei sechsj\"ahrige Zwillinge haben. Als die Kinder zusammen im Garten spielen, darf Ir�ne nicht ins Baumhaus, weil dort keine ``Dicken'' rein d\"urfen. Vom Baumhaus aus rufen die Zwillinge zu Ir�ne herab, ob sie kein eigenes Haus habe und lachen dabei.
   
   
   \section{CV}
\label{cv}
\addcontentsline{toc}{section}{Appendix~\ref{cv}: CV}
   
\textbf{Personalien:}\\
Damian Hiltebrand\\
N\"orgelbach 4\\
8493 Saland\\
\\
E-Mail: damian.h82@gmail.com\\
Geburtsdatum: 8.10.1982\\
Telefon: 078 823 63 76\\
\\
\hspace{-0.95cm}\textbf{Ausbildung:}
\begin{itemize}
\item 2003-2013: Studium Universi\"at Z\"urich, Hauptfach: Psychologie, 1. Nebenfach: Computerlinguistik, 2. Nebenfach: Informatik
\item{1995-2002: Matura an Kantonsschule Rychenberg in Winterthur, Typus B}
\item{1989-1995: Primarschule in Weisslingen und Saland}
\end{itemize}
\vspace{1cm}
\textbf{T\"atigkeiten w\"ahrend des Studiums:}
\begin{itemize} 
\item Tutorate: Einf\"uhrung in die Computerlinguistik II (3 Semester) und IPS (Interaktives Proseminar) (1 Semester).
\item Nebenerwerb ab 2006: Auswerten von Daten und Verfassen von Artikeln im ISGF (Institut f\"ur Sucht- und Gesundheitsforschung) f�r das BAG (Bundesamt f�r Gesundheit)
\item Interviewer f�r GIM (Gesellschaft f�r Innovative Marktforschung)
\item Selbstst\"andiges Aufbauen einer Computerspiel Engine mit OpenGL und XNA
\end{itemize}

  % \section{Additional graphics}
 %  \section{Perception of Laughter Test -- Items}
  % \section{Ridicule Teasing Scenario questionnaire -- Items}


   \end{document}
   